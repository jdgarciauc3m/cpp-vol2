\section{Destructores}

\begin{frame}[fragile]{El problema de la desasignación}
\begin{itemize}
  \item Un tipo que adquiere \textmark{recursos} (ej. memoria) 
        y no lo libera provoca un \textbad{goteo de memoria}.

  \mode<presentation>{\vfill\pause}
  \item \textgood{Solución simple}:
    \begin{itemize}
      \item Función miembro de liberación.
    \end{itemize}
\begin{lstlisting}
class vecnum {
  // ...
  void libera() { delete []v; }
  // ...
};

void f() {
  vecnum v{5};
  // ...
  otra_funcion(); // <- ¿Y si lanza una excepción?
  // ...
  v.libera(); // Fácil de olvidar
}
\end{lstlisting}
\end{itemize}
\end{frame}

\begin{frame}[fragile]{Función miembro de destrucción}
\begin{itemize}
  \item Un \textgood{destructor} es una función miembro especial que se ejecuta 
        \textmark{de forma automática} cuando un objeto sale de alcance.
    \begin{itemize}
      \item No tiene tipo de retorno.
      \item No toma parámetros.
      \item Su nombre es el nombre de clase precedido del símbolo \cppkey{\~}.
    \end{itemize}

\mode<presentation>{\vfill\pause}
\begin{lstlisting}
class vector {
  // ...
  ~vector() { delete []v; }
  // ...
};
\end{lstlisting}

  \mode<presentation>{\vfill\pause}
  \item Invocación automática.
\begin{lstlisting}
void f() {
  vecnum v{5};
  // ...
  otra_funcion(); // Si lanza excepción -> v se destruye
  // ...
} // v se destruye al salir
\end{lstlisting}
\end{itemize}
\end{frame}

\begin{frame}[fragile]{Destrucción generada por defecto}
\begin{itemize}
  \item En realidad \textmark{todos los tipos} tienen una función miembro 
        \textgood{destructor}.
    \begin{itemize}
      \item El \textmark{compilador genera automáticamente} una definición 
            que es \textgood{válida} en la mayoría de los casos.
    \end{itemize}

  \mode<presentation>{\vfill\pause}
  \item Dado un tipo \textbad{sin destructor}, 
        el compilador \textgood{sintetiza} uno que:
    \begin{itemize}
      \item Invoca (recursivamente) al destructor de cada miembro.
      \item El destructor de un tipo primitivo es la operación nula.
    \end{itemize}
\end{itemize}

\mode<presentation>{\vfill\pause}
\begin{lstlisting}
struct estudiante {
  string nombre;
  vector<double> notas;
  // ...
};

void f() {
  estudiante e { "Carlos", {9.0, 9.5, 8.5} };
  cout << e.nombre << " -> " << e.notas[0] << endl;
} // Destrucción de e -> Destrucción de e.nombre y e.notas
\end{lstlisting}
\end{frame}

\begin{frame}[t]{Vector numérico con destructor}
\begin{block}{vecnum.hpp}
\mode<presentation>{
  \lstinputlisting[basicstyle=\tiny]{ejemplos/01-copy-move/primitive-vecnum-destr/vecnum.hpp}
}
\mode<article>{
  \lstinputlisting{ejemplos/01-copy-move/primitive-vecnum-destr/vecnum.hpp}
}
\end{block}
\end{frame}
