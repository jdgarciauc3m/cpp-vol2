\section{Expresiones lambda}

\subsection{¿Qué es una expresión lambda?}

\defverbatim[colored]\lambdaeja{
\begin{lstlisting}
class menor_abs {
  bool operator()(int x, int y) { return abs(x) < abs(y); }
};

//...
std::vector<int> v = { 100, -1, 40, -70 };
std::sort(v.begin(), v.end(), menor_abs());
\end{lstlisting}
}
\defverbatim[colored]\lambdaejb{
\begin{lstlisting}
std::vector<int> v = { 100, -1, 40, -70 };
std::sort(v.begin(), v.end(), 
          [](int x, int y) { return abs(x) < abs(y); });
\end{lstlisting}}
\begin{frame}
  \begin{itemize}
    \item Una expresión lambda:
      \begin{itemize}
        \item Es un mecanismo para especificar un objeto función.
        \item Permite pasar una función simple a otro contexto.
      \end{itemize}
    \item \pause C++03:
  \end{itemize}
\lambdaeja
  \begin{itemize}
    \item \pause C++11:
  \end{itemize}
\lambdaejb
\end{frame}

\defverbatim[colored]\lambdapartes{
\begin{lstlisting}
std::sort(v.begin(), v.end(), 
  [](int x, int y) 
  { 
    return abs(x) < abs(y); 
  }
);
\end{lstlisting}}
\begin{frame}{Partes de una lambda}
\lambdapartes
  \begin{itemize}
    \item \pause \textmark{Introductor de lambda}: Indica el comienzo de una expresión lambda.
    \item \pause \textmark{Parámetros de lambda}: Equivalente a parámetros de una función.
    \item \pause \textmark{Cuerpo de la función lambda}: Equivalente a cuerpo de una función.
    \item \pause \textmark{Tipo de retorno}: Se puede deducir automáticamente.
  \end{itemize}
\end{frame}

\subsection{Capturas}

\begin{frame}[t]{A qué variables puede acceder una lambda}
  \begin{itemize}
    \item Acceso por defecto:
      \begin{itemize}
        \item Variables globales.
        \item Variables estáticas locales.
      \end{itemize}
    \mode<presentation>{\vfill\pause}
    \item Acceso que necesita declaración explícita:
      \begin{itemize}
        \item Captura por defecto: Puede ser por referencia o valor.
        \item Captura explícita de variable: Puede ser por referencia o valor.
        \item Captura de \cppkey{this}.
      \end{itemize}
  \end{itemize}
\end{frame}

\begin{frame}[t,fragile]{Acceso por defecto}
\begin{lstlisting}[escapechar=@]
double s { 0.0 };

void f(int i) {
  char c {'a'};
  static int n {};

  vector<int> v { 1, 2, 3};
  for_each(v.begin(), v.end(),
    [](int x) { cout << x; }); // OK@\pause@
  for_each(v.begin(), v.end(),
    [](int x) { cout << x << s; }); // OK. s es global@\pause@
  for_each(v.begin(), v.end(),
    [](int x) { cout << x << i; }); // Error. i no accesible@\pause@
  for_each(v.begin(), v.end(),
    [](int x) { cout << x << c; }); // Error. c no accesible@\pause@
  for_each(v.begin(), v.end(),
    [](int x) { cout << x << n; }); // OK. n es estático
}
\end{lstlisting}
\end{frame}

\begin{frame}[t,fragile]{Captura explícita por valor}
  \begin{itemize}
    \item Se enumeran los nombres de las variables en la lista de captura.
  \end{itemize}
\pause
\begin{lstlisting}[escapechar=@]
void imprime_rango(std::vector<int> & v, int minv, int maxv) {
  std::for_each(v.begin(), v.end(),
    [minv,maxv](int x) { 
       if (x > maxv) return;
       if (x < minv) return;
       cout << x << endl;
  });
}
\end{lstlisting}
\end{frame}

\begin{frame}[t,fragile]{Captura ímplicita por valor}
  \begin{itemize}
    \item Se indica con \cppid{=} en la lista de captura.
    \item Se capturan por valor todos los objetos en el ámbito contenedor.
  \end{itemize}
\pause
\begin{lstlisting}[escapechar=@]
void imprime_rango(std::vector<int> & v, int minv, int maxv) {
  std::for_each(v.begin(), v.end(),
    [=](int x) { 
       if (x > maxv) return;
       if (x < minv) return;
       cout << x << endl;
  });
}
\end{lstlisting}
\end{frame}

\begin{frame}[t,fragile]{Captura por valor}
  \begin{itemize}
    \item ¿Se copian todos los valores al objeto captura?
      \begin{itemize}
        \item \pause No. Solamente los que efectivamente se usan dentro de la lambda.
        \item La captura solamente indica accesibilidad.
      \end{itemize}
  \end{itemize}
\pause
\begin{lstlisting}[escapechar=@]
void imprime_rango(std::vector<int> & v, int minv, int maxv) {
  int w[512];
  std::for_each(v.begin(), v.end(),
    [=](int x) { // Atención: w no se copia.
       if (x > maxv) return;
       if (x < minv) return;
       cout << x << endl;
  });
}
\end{lstlisting}
\end{frame}

\begin{frame}[t,fragile]{Captura explícita por referencia}
  \begin{itemize}
    \item Se enumeran los nombres de las variables en la lista de captura, anteponiéndoles \cppid{\&}.
    \item Se pueden combinar capturas por valor y referencia.
    \item No existe la idea de captura por referencia constante.
  \end{itemize}
\pause
\begin{lstlisting}[escapechar=@]
int suma_rango(std::vector<int> & v, int minv, int maxv) {
  int suma { 0 };
  std::for_each(v.begin(), v.end(),
    [&suma, minv, maxv](int x) {
       if (x > maxv) return;
       if (x < minv) return;
       suma += x;
  });
  return suma;
}
\end{lstlisting}
\end{frame}

\begin{frame}[t,fragile]{Captura implícita por referencia}
  \begin{itemize}
    \item Se indica con \cppid{\&} en la lista de captura.
    \item Se capturan por referencia todos los objetos en el ámbito contenedor.
  \end{itemize}
\pause
\begin{lstlisting}[escapechar=@]
int suma_rango(std::vector<int> & v, int minv, int maxv) {
  int suma { 0 };
  std::for_each(v.begin(), v.end(),
    [&](int x) {
       if (x > maxv) return;
       if (x < minv) return;
       suma += x;
  });
  return suma;
}
\end{lstlisting}
\end{frame}

\begin{frame}[t,fragile]{Combinación de capturas}
  \begin{itemize}
    \item Se puede indicar una captura implícita seguida de capturas explícitas.
  \end{itemize}
\pause
\begin{lstlisting}
int suma_rango(std::vector<int> & v, int minv, int maxv) {
  int suma { 0 };
  std::for_each(v.begin(), v.end(),
    [=,&suma](int x) {
       if (x > maxv) return;
       if (x < minv) return;
       suma += x;
  });
  return suma;
}
\end{lstlisting}
\end{frame}

\begin{frame}[t,fragile]{Acceso a miembros desde una lambda}
  \begin{itemize}
    \item Desde una lambda en una función miembro no se pueden capturar
    explícitamente datos miembro.
  \end{itemize}
\begin{lstlisting}
struct X {
  std::vector<int> v;
  int minv;
  int maxv;
  int suma_rango() {
    int suma { 0 };
    std::for_each(v.begin(), v.end(), 
      [minv,maxv,&suma](int x) { // Error
         if (x > maxv) return;
         if (x < minv) return;
         suma += x;
    });
    return suma;
  }
};
\end{lstlisting}
\end{frame}
 
\defverbatim[colored]\lambdathisuno{
\begin{lstlisting}
struct X {
  std::vector<int> v;
  int minv;
  int maxv;
  int suma_rango() {
    int suma { 0 };
    std::for_each(v.begin(), v.end(), 
      [=,&suma](int x) {
         if (x > maxv) return;
         if (x < minv) return;
         suma += x;
    });
    return suma;
  }
};
\end{lstlisting}}
\defverbatim[colored]\lambdathisdos{
\begin{lstlisting}[escapechar=@]
struct X {
  std::vector<int> v;
  int minv;
  int maxv;
  int suma_rango() {
    int suma { 0 };
    std::for_each(v.begin(), v.end(), 
      [&](int x) {
         if (x > maxv) return;
         if (x < minv) return;
         suma += x;
    });
    return suma;
  }
};
\end{lstlisting}}
\defverbatim[colored]\lambdathistres{
\begin{lstlisting}[escapechar=@]
struct X {
  std::vector<int> v;
  int minv;
  int maxv;
  int suma_rango() {
    int suma { 0 };
    std::for_each(v.begin(), v.end(), 
      [this,&suma](int x) {
         if (x > maxv) return;
         if (x < minv) return;
         suma += x;
    });
    return suma;
  }
};
\end{lstlisting}}
\begin{frame}{Acceso a miembros desde una lambda}
  \begin{itemize}
    \item Si se puede capturar implícitamente el objeto\only<1>{\alert{ por valor}}\only<2>{\alert{ por referencia}}.
    \item <3->Se puede capturar explícitamente \cppkey{this}.
  \end{itemize}
\only<1>{\lambdathisuno}\only<2>{\lambdathisdos}\only<3>{\lambdathistres}
\end{frame}

\subsection{Parámetros}

\begin{frame}[t,fragile]{Omisión de parámetros}
  \begin{itemize}
    \item Si la lambda no toma parámetros, estos se pueden omitir.
  \end{itemize}
\begin{lstlisting}
std::thread([](){ haz_algo(); });

std::thread([]{ haz_algo(); });
\end{lstlisting}
\end{frame}

\subsection{Tipo de retorno}

\begin{frame}[t,fragile]{Especificación de tipo de retorno}
  \begin{itemize}
    \item Se puede especificar explícitamente el tipo de retorno de una lambda.
    \item Se usa la nueva especificación postfija de tipo de retorno.
  \end{itemize}
\pause
\begin{lstlisting}
std::sort(v.begin(), v.end(), 
  [](int x, int y) -> bool
  { 
    return abs(x) < abs(y); 
  }
);
\end{lstlisting}
\end{frame}

\begin{frame}[t,fragile]{Omisión de tipo de retorno}
  \begin{itemize}
    \item Se puede deducir el tipo de retorno, cuando el cuerpo es de la forma:
  \end{itemize}
\begin{lstlisting}
{ return expresion; }
\end{lstlisting}
\pause
  \begin{itemize}
    \item Reglas:
    \begin{itemize}
      \item El tipo de retorno es el tipo de la expresión.
      \item Se aplican si es necesario conversiones \emph{l-valor} a \emph{r-valor}, \emph{array} a puntero, y función a puntero.
    \end{itemize}
  \end{itemize}
\begin{lstlisting}
[&] { return x+y; }
[&] () -> decltype(x+y) { return x+y; }
\end{lstlisting}
\pause
  \begin{itemize}
    \item En otro caso el tipo de retorno se considera \cppkey{void}.
  \end{itemize}
\begin{lstlisting}
[] { haz_algo(); }
[] () -> void { haz_algo(); }
\end{lstlisting}
\end{frame}

\subsection{Modificadores}

\begin{frame}[t,fragile]{Mutabilidad de una lambda}
  \begin{itemize}
    \item Por defecto una lambda da lugar a un operador de llamada \cppkey{const} en el objeto función.
    \item Si se especifica que la lambda es \cppkey{mutable}, da lugar a un objeto función que es no constante.
  \end{itemize}
\begin{lstlisting}
int ini=0;
std::generate(v.begin(), v.end(), [=]() mutable{ return ini++; }
std::generate(w.begin(), w.end(), [=]() mutable -> int { return ini++; }
\end{lstlisting}
\end{frame}

\subsection{Lambdas y objetos clausura}

\begin{frame}[t,fragile]{Objeto clausura}
  \begin{itemize}
    \item La evaluación de una expresión lambda da como resultado un temporal llamado \emph{objeto clausura}.
    \item Un objeto clausura se comporta como un objeto función.
    \item El tipo de una expresión lambda es el mismo que el tipo de su objeto clausura:
      \begin{itemize}
        \item Tipo único.
        \item Tipo sin nombre.
        \item Tipo de clase que no es unión.
        \item No es un agregado.
        \item Declarado en el alcance más pequeño que contiene la expresión lambda.
      \end{itemize}
  \end{itemize}
\begin{lstlisting}
auto comp = [](int x, int y) { return x>y; }
sort(v1.begin(), v1.end(), comp);
bool test = comp(2,4);
\end{lstlisting}
\end{frame}

\begin{frame}{Miembros generados de una clausura}
  \begin{itemize}
    \item Tiene un constructor por defecto borrado.
    \item Tiene un operador de asignación de copia borrado.
    \item Tiene un constructor de copia implícitamente declarado.
    \item Puede tener un constructor de movimiento implícitamente declarado.
    \item Tiene un destructor implícitamente declarado.
    \item Tiene un operador de llamada a función (\cppkey{operator()}) que es \cppkey{inline}.
      \begin{itemize}
        \item Parámetros y tipo de retorno coinciden con lambda.
        \item El operador es \cppkey{const} si la lambda se declaró sin modificador \cppkey{mutable}.
      \end{itemize}
  \end{itemize}
\end{frame}

\begin{frame}[t,fragile]{Conversión a puntero a función}
  \begin{itemize}
    \item Condición:
      \begin{itemize}
        \item Debe tener una captura vacía.
      \end{itemize}
    \item Operador de conversión generado:
      \begin{itemize}
        \item Conversión a puntero a función.
        \item Operador de conversión público, no virtual, no explícito, constante.
        \item Mismos parámetros y tipo de retorno.
        \item Devuelve un puntero a una función cuya invocación tiene el mismo efecto que invocar al 
        operador de llamada a función del objeto clausura.
      \end{itemize}
  \end{itemize}
\begin{lstlisting}
void set_func(void (*f)());
set_func([]{ std::cerr << "Adios"; });
\end{lstlisting}
\end{frame}

\begin{frame}[t,fragile]{Peligro de la captura por referencia}
  \begin{itemize}
    \item La captura por referencia puede proporcionar beneficios, pero ...
    \item \pause Presenta peligros si la lambda sobrevive a un valor capturado.
      \begin{itemize}
        \item Ejemplo: Callbacks.
      \end{itemize}
  \end{itemize}
\begin{lstlisting}
void f() {
  int suma = 0;
  registra_sumador([&suma](int x) { suma+=x; });
  // ...
}
\end{lstlisting}
\end{frame}
