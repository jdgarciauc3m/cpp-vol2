\section{Plantillas con número variable de argumentos}

\subsection{Necesidad de plantillas variables}

\begin{frame}[fragile]{Plantilla y número de argumentos}
  \begin{itemize}
    \item Se pueden definir plantillas con un número variable de argumentos.
    \item Motivación:
      \begin{itemize}
        \item \pause Poder definir un tipo \cppid{tupla} (de cualquier dimensión).
\begin{lstlisting}
tupla<int,char> t1;
tupla<int,long,int,string> t2;
\end{lstlisting}
        \item \pause Poder definir funciones con número variable de parámetros que
        sean fuertemente tipadas.
\begin{lstlisting}
mi_print("Hola %: naciste en %", "Daniel", 1969); // No hay especificador de formato
\end{lstlisting}
      \end{itemize}
    \item \pause El uso de plantillas requiere un uso intensivo de recursividad.
  \end{itemize}
\end{frame}

\subsection{Sintaxis y operaciones}

\begin{frame}[fragile]{Declaración}
  \begin{itemize}
    \item Clases
\begin{lstlisting}
template <typename ... Args>
class tupla;
\end{lstlisting}
    \item \pause Funciones
\begin{lstlisting}
void mi_print(const char *s);

template <typename T, typename ... Args>
void mi_print(const char * s, T v, Args ... args);
\end{lstlisting}
    \item \pause Argumentos no-tipo.
\begin{lstlisting}
template <typename T, std::size_t ... D>
class MultiVec;
\end{lstlisting}
  \end{itemize}
\end{frame}

\begin{frame}[fragile]{Paquetes de parámetros}
  \begin{itemize}
    \item Paquetes de parámetros de plantillas.
\begin{lstlisting}
template <class ... Args>
class tupla { /*...*/ };

tupla<int,long,int,string> t2; // Args=int,long,int,string
\end{lstlisting}
    \item \pause Paquete de parámetros de función.
\begin{lstlisting}
void mi_print(const char *s);

template <typename T, typename ... Args> 
void mi_print(const char * s, T v, Args ... args); 
\end{lstlisting}
    \begin{itemize}
      \item Args $\rightarrow$ Paquete de parámetros de plantilla.
      \item args $\rightarrow$ Paquete de parámteros de función.
    \end{itemize}
\begin{lstlisting}
mi_print("Hola %: naciste en %", "Daniel", 1969); 
\end{lstlisting}
  \end{itemize}
\end{frame}

\begin{frame}[t]{Operaciones sobre paquetes}
  \begin{itemize}
    \item Solamente dos operaciones.
      \begin{itemize}
        \item \pause Desempaquetar o expandir: \cppid{...}.
          \begin{itemize}
            \item Aplicable a paquetes de parámetros de plantilla y a paquetes
            de parámetros de función.
            \item \cppid{Args...} se expande a una lista de argumentos.
          \end{itemize}
        \item \pause Contar el número delementos: \cppkey{sizeof}\cppid{...()}
          \begin{itemize}
            \item Determina el número de elementos en un paquete.
            \item \cppkey{sizeof}\cppid{...(Args)}.
          \end{itemize}
      \end{itemize}
  \end{itemize}
\end{frame}

\subsection{Funciones}

\begin{frame}[fragile]
\begin{lstlisting}
void mi_print(const char * s) {
  if (s==nullptr) return;
  while (*s) {
    if (*s=='%' and *++s!='%')
      throw runtime_error("Falta argumento");
    cout << *s++;
  }
}

template <typename T, typename ... Args>
void mi_print(const char * s, T v, Args ... args) {
  if (s==nullptr) return;
  while (*s) {
    if (*s=='%' && *++s!='%') {
      cout << v;
      mi_print(++s, args...);
      return;
    }
    cout << *s++;
  }
  throw runtime_error("Demasiados argumentos");
}
\end{lstlisting}
\end{frame}

\subsection{Clases}

\begin{frame}[fragile]
\begin{lstlisting}
template <typename ... Args>
class tupla;

template <> class tupla<> {}; // Tupla vacía

template <typename T, typename ... Args>
class tupla<T,Args...> : private tupla<Args...> {
private:
  T dato;
public:
  tupla() : dato{} {}
  /* ... */	
};
\end{lstlisting}
\begin{itemize}
  \item \pause \cppid{tupla<int,long,int,string>}
  \item \cppid{tupla<long,int,string>}
  \item \cppid{tupla<int,string>}
  \item \cppid{tupla<string>}
  \item \cppid{tupla<>}
\end{itemize}
\end{frame}

\begin{frame}[fragile]
\begin{lstlisting}
template <std::size_t N, typename T>
struct elemento;

template <typename T, typename ... Args>
struct elemento<0,tupla<T,Args...>> {
  using tipo = T;
};

template <std::size_t N, typename T, typename ... Args>
struct elemento<N, tupla<T,Args...>> {
  using tipo = typename elemento<N-1, tupla<Args...>>::tipo;
};
\end{lstlisting}
\end{frame}
