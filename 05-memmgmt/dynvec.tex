\section{Un vector dinámico con \textbf{unique\_ptr}}

\begin{frame}[t,fragile]{Vector dinámico}
\begin{itemize}
  \item \textmark{Objetivo}: Permitir que se puedan añadir elementos.
\begin{lstlisting}
vecnum v(10); // 10 elementos
v.agrega_final(1.5); // 11 elementos
v.agrega_final(2.5); // 12 elementos
\end{lstlisting}

  \mode<presentation>{\vfill\pause}
  \item \textbad{Problema}: Cuando \cppid{v} se crea tiene espacio para \cppid{10} elementos.

  \mode<presentation>{\vfill\pause}
  \item \textgood{Estrategia}:
    \begin{itemize}
      \item \textmark{Diferenciar} entre:
        \begin{itemize}
          \item \textemph{Tamaño}: Número de elementos que tiene el vector
          \item \textemph{Capacidad}: Número máximo de elementos que caben en el búfer.
        \end{itemize}
      \item Cuando el \textbad{vector se llena} $\Rightarrow$
            Reservar un \textgood{búfer mayor} y copiar los elementos.
    \end{itemize}
\end{itemize}
\end{frame}

\begin{frame}[t]{Operaciones sobre tamaño y capacidad}
\begin{itemize}
  \item \cppid{reserva(n)}: Asegura que hay espacio para \cppid{n} elementos.
    \begin{itemize}
      \item Hace que la \textmark{capacidad} sea al menos \cppid{n}.
      \item Si \cppid{n} es menor o igual que la \cppid{capacidad} no se hace nada.
    \end{itemize}

  \mode<presentation>{\vfill\pause}
  \item \cppid{redimensiona(n)}: Cambia el tamaño a exactamente \cppid{n} elementos.
    \begin{itemize}
      \item Redimensiona para que haya \cppid{n} elementos completando con
            valores a \cppid{0.0} si es necesario.
    \end{itemize}

  \mode<presentation>{\vfill\pause}
  \item \cppid{agrega\_final(x)}: Agrega el valor \cppid{x} al final del vector,
        incrementando el \textmark{tamaño}. 
    \begin{itemize}
      \item Incrementa el valor del \textmark{tamaño}.
      \item Si el búfer esta completo, llama a \cppid{redimensiona()} con una
            \textmark{capacidad} mayor.
    \end{itemize}
\end{itemize}
\end{frame}

\begin{frame}[t,fragile]{Definiendo \textbf{vecnum}}
\begin{block}{vecnum.hpp}
\begin{lstlisting}
#ifndef VECNUM_VECNUM_HPP
#define VECNUM_VECNUM_HPP

#include <memory>

class vecnum {
  //...
private:
  int tamanyo_ = 0;
  int capacidad_ = 0;
  std::unique_ptr<double[]> buffer_ = nullptr;
};

#endif// VECNUM_VECNUM_HPP
\end{lstlisting}
\end{block}
\end{frame}

\begin{frame}[t,fragile]{Constructores de \textbf{vecnum}}
\begin{block}{vecnum.hpp}
\begin{lstlisting}
class vecnum {
public:
  vecnum() noexcept = default;
  explicit vecnum(int n)
      : capacidad_{std::max(0, n)},
        buffer_{std::make_unique<double[]>(capacidad_)} {}
  vecnum(std::initializer_list<double> l);

  vecnum(const vecnum & v);
  vecnum & operator=(const vecnum & v);

  vecnum(vecnum &&) noexcept = default;
  vecnum & operator=(vecnum &&) noexcept = default;

  ~vecnum() noexcept = default;
//...
};
\end{lstlisting}
\end{block}
\end{frame}

\begin{frame}[t,fragile]{Constructor de lista de \textbf{vecnum}}
\begin{block}{vecnum.cpp}
\begin{lstlisting}
vecnum::vecnum(std::initializer_list<double> l)
    : tamanyo_{static_cast<int>(l.size())},
      capacidad_{tamanyo_},
      buffer_{std::make_unique_for_overwrite<double[]>(capacidad_)} {
  std::copy_n(l.begin(), l.size(), buffer_.get());
}
\end{lstlisting}
\end{block}
\end{frame}

\begin{frame}[t,fragile]{Operaciones de copia de \textbf{vecnum}}
\begin{block}{vecnum.cpp}
\begin{lstlisting}
vecnum::vecnum(const vecnum & v)
    : tamanyo_{v.tamanyo_},
      capacidad_{v.capacidad_},
      buffer_{std::make_unique_for_overwrite<double[]>(capacidad_)} {
  std::copy_n(v.buffer_.get(), v.tamanyo_, buffer_.get());
}

vecnum & vecnum::operator=(const vecnum & v) {
  if (this == &v) { return *this; }
  auto aux = std::make_unique_for_overwrite<double[]>(v.capacidad_);
  std::copy_n(v.buffer_.get(), v.tamanyo_, aux.get());
  tamanyo_ = v.tamanyo_;
  capacidad_ = v.capacidad_;
  buffer_ = std::move(aux);
  return *this;
}
\end{lstlisting}
\end{block}
\end{frame}

\begin{frame}[t,fragile]{Otras operaciones de \textbf{vecnum}}
\begin{block}{vecnum.hpp}
\begin{lstlisting}
class vecnum {
public:
  //...
  void reserva(int n);
  void redimensiona(int n);
  void agrega_final(double x);

  [[nodiscard]] double operator[](int i) const noexcept { return buffer_[i]; }
  [[nodiscard]] double & operator[](int i) noexcept { return buffer_[i]; }

  [[nodiscard]] int tamanyo() const noexcept { return tamanyo_; }
  [[nodiscard]] int capacidad() const noexcept { return capacidad_; }
  //...
};
\end{lstlisting}
\end{block}
\end{frame}

\begin{frame}[t]{Reserva}
\begin{itemize}
  \item \cppid{reserva(n)}: Asegura la reserva espacio para \cppid{n} elementos.
    \begin{itemize}
      \item Si \cppid{n} es menor que el espacio actual no hace nada.
      \item Deja los elementos nuevos sin iniciar.
    \end{itemize}
\end{itemize}

\mode<presentation>{\vfill\pause}
\begin{tikzpicture}
\tikzset{
    bloque/.style={rectangle,draw=black, top color=white, bottom color=blue!50,
                   very thick, inner sep=0.5em, minimum size=0.6cm, text centered, font=\tiny},
    bloquelibre/.style={rectangle,draw=black, top color=white, bottom color=blue!10,
                   very thick, inner sep=0.5em, minimum size=0.6cm, text centered, font=\tiny},
    flecha/.style={->, >=latex', shorten >=1pt, thick},
    etiqueta/.style={text centered, font=\tiny} 
}  
\node[bloque] (bsize) {2};
\node[bloque, right=0cm of bsize] (bcap) {4};
\node[bloque,right=0cm of bcap] (bptr) { };
\node[bloque,right=0.5cm of bptr] (v0) {1.0};
\node[bloque,right=0cm of v0] (v1) {2.0};
\node[bloquelibre,right=0cm of v1] (v2) {?};
\node[bloquelibre,right=0cm of v2] (v3) {?};
\draw[flecha] (bptr) -- (v0);
\node[etiqueta, left=0.1cm of bsize] {v1:};
\node[etiqueta, above=0cm of bsize] {tam};
\node[etiqueta, above=0cm of bcap] {cap};
\node[etiqueta, above=0cm of bptr] {vec};
\end{tikzpicture}

\mode<presentation>{\vfill\pause}
\begin{itemize}
  \item \cppid{v.reserva(8)}
\end{itemize}

\begin{tikzpicture}
\tikzset{
    bloque/.style={rectangle,draw=black, top color=white, bottom color=blue!50,
                   very thick, inner sep=0.5em, minimum size=0.6cm, text centered, font=\tiny},
    bloquelibre/.style={rectangle,draw=black, top color=white, bottom color=blue!10,
                   very thick, inner sep=0.5em, minimum size=0.6cm, text centered, font=\tiny},
    flecha/.style={->, >=latex', shorten >=1pt, thick},
    etiqueta/.style={text centered, font=\tiny} 
}  
\node[bloque] (bsize) {2};
\node[bloque, right=0cm of bsize] (bcap) {8};
\node[bloque,right=0cm of bcap] (bptr) { };

\node[bloque,right=0.5cm of bptr] (v0) {1.0};
\node[bloque,right=0cm of v0] (v1) {2.0};
\node[bloquelibre,right=0cm of v1] (v2) {?};
\node[bloquelibre,right=0cm of v2] (v3) {?};
\node[etiqueta, right=0.5cm of v3] (delete) {\cppkey{delete[]}};
\draw[flecha] (delete) -- (v3);

\node[bloque, below right=0.5cm and 0.5cm of bptr] (nv0) {1.0};
\node[bloque,right=0cm of nv0] (nv1) {2.0};
\node[bloquelibre,right=0cm of nv1] (nv2) {?};
\node[bloquelibre,right=0cm of nv2] (nv3) {?};
\node[bloquelibre,right=0cm of nv3] (nv4) {?};
\node[bloquelibre,right=0cm of nv4] (nv5) {?};
\node[bloquelibre,right=0cm of nv5] (nv6) {?};
\node[bloquelibre,right=0cm of nv6] (nv7) {?};
\node[etiqueta, right=0.5cm of nv7] (new) {\cppkey{new[]}};
\draw[flecha] (new) -- (nv7);

\draw[flecha] (bptr) -- (nv0);
\node[etiqueta, left=0.1cm of bsize] {v1:};
\node[etiqueta, above=0cm of bsize] {tam};
\node[etiqueta, above=0cm of bcap] {cap};
\node[etiqueta, above=0cm of bptr] {vec};
\end{tikzpicture}
\end{frame}

\begin{frame}[t,fragile]{Reserva}
\begin{block}{vecnum.cpp}
\begin{lstlisting}
void vecnum::reserva(int n) {
  if (n > capacidad_) {
    auto aux = std::make_unique_for_overwrite<double[]>(n);
    std::copy_n(buffer_.get(), tamanyo_, aux.get());
    std::swap(buffer_, aux);
    capacidad_ = n;
  }
}
\end{lstlisting}
\end{block}
\end{frame}

\begin{frame}[t]{Cambio de tamaño}
\begin{itemize}
  \item \cppid{redimiensiona(n)}: Cambia el tamaño de un vector.
    \begin{itemize}
      \item \cppid{n} excede \textmark{capacidad} total $\rightarrow$ \cppid{reserva(n)}
      \item \cppid{n} mayor que \textmark{tamaño} actual 
            pero no excede \textmark{capacidad} total $\rightarrow$ 
            Iniciar \textmark{elementos} nuevos a \cppid{0} y actualizar \textmark{tamaño}.
      \item \cppid{n} menor que \textmark{tamaño} actual $\rightarrow$ 
            Actualizar \textmark{tamaño}.
    \end{itemize}
\end{itemize}

\mode<presentation>{\vfill\pause}
\begin{tikzpicture}
\tikzset{
    bloque/.style={rectangle,draw=black, top color=white, bottom color=blue!50,
                   very thick, inner sep=0.5em, minimum size=0.6cm, text centered, font=\tiny},
    bloquelibre/.style={rectangle,draw=black, top color=white, bottom color=blue!10,
                   very thick, inner sep=0.5em, minimum size=0.6cm, text centered, font=\tiny},
    flecha/.style={->, >=latex', shorten >=1pt, thick},
    etiqueta/.style={text centered, font=\tiny} 
}  
\node[bloque] (bsize) {2};
\node[bloque, right=0cm of bsize] (bcap) {4};
\node[bloque,right=0cm of bcap] (bptr) { };
\node[bloque,right=0.5cm of bptr] (v0) {1.0};
\node[bloque,right=0cm of v0] (v1) {2.0};
\node[bloquelibre,right=0cm of v1] (v2) {?};
\node[bloquelibre,right=0cm of v2] (v3) {?};
\draw[flecha] (bptr) -- (v0);
\node[etiqueta, left=0.1cm of bsize] {v1:};
\node[etiqueta, above=0cm of bsize] {tam};
\node[etiqueta, above=0cm of bcap] {cap};
\node[etiqueta, above=0cm of bptr] {vec};
\end{tikzpicture}

\mode<presentation>{\vfill\pause}
\begin{itemize}
  \item \cppid{redimensiona(3)}
\end{itemize}
\pause
\begin{tikzpicture}
\tikzset{
    bloque/.style={rectangle,draw=black, top color=white, bottom color=blue!50,
                   very thick, inner sep=0.5em, minimum size=0.6cm, text centered, font=\tiny},
    bloquelibre/.style={rectangle,draw=black, top color=white, bottom color=blue!10,
                   very thick, inner sep=0.5em, minimum size=0.6cm, text centered, font=\tiny},
    flecha/.style={->, >=latex', shorten >=1pt, thick},
    etiqueta/.style={text centered, font=\tiny} 
}  
\node[bloque] (bsize) {3};
\node[bloque, right=0cm of bsize] (bcap) {4};
\node[bloque,right=0cm of bcap] (bptr) { };
\node[bloque,right=0.5cm of bptr] (v0) {1.0};
\node[bloque,right=0cm of v0] (v1) {2.0};
\node[bloque,right=0cm of v1] (v2) {0.0};
\node[bloquelibre,right=0cm of v2] (v3) {?};
\draw[flecha] (bptr) -- (v0);
\node[etiqueta, left=0.1cm of bsize] {v1:};
\node[etiqueta, above=0cm of bsize] {tam};
\node[etiqueta, above=0cm of bcap] {cap};
\node[etiqueta, above=0cm of bptr] {vec};
\end{tikzpicture}
\end{frame}

\begin{frame}[t]{Cambio de tamaño}
\begin{tikzpicture}
\tikzset{
    bloque/.style={rectangle,draw=black, top color=white, bottom color=blue!50,
                   very thick, inner sep=0.5em, minimum size=0.6cm, text centered, font=\tiny},
    bloquelibre/.style={rectangle,draw=black, top color=white, bottom color=blue!10,
                   very thick, inner sep=0.5em, minimum size=0.6cm, text centered, font=\tiny},
    flecha/.style={->, >=latex', shorten >=1pt, thick},
    etiqueta/.style={text centered, font=\tiny} 
}  
\node[bloque] (bsize) {3};
\node[bloque, right=0cm of bsize] (bcap) {4};
\node[bloque,right=0cm of bcap] (bptr) { };
\node[bloque,right=0.5cm of bptr] (v0) {1.0};
\node[bloque,right=0cm of v0] (v1) {2.0};
\node[bloque,right=0cm of v1] (v2) {0.0};
\node[bloquelibre,right=0cm of v2] (v3) {?};
\draw[flecha] (bptr) -- (v0);
\node[etiqueta, left=0.1cm of bsize] {v1:};
\node[etiqueta, above=0cm of bsize] {tam};
\node[etiqueta, above=0cm of bcap] {cap};
\node[etiqueta, above=0cm of bptr] {vec};
\end{tikzpicture}

\mode<presentation>{\vfill\pause}
\begin{itemize}
  \item \cppid{redimensiona(1)}
\end{itemize}
\pause
\begin{tikzpicture}
\tikzset{
    bloque/.style={rectangle,draw=black, top color=white, bottom color=blue!50,
                   very thick, inner sep=0.5em, minimum size=0.6cm, text centered, font=\tiny},
    bloquelibre/.style={rectangle,draw=black, top color=white, bottom color=blue!10,
                   very thick, inner sep=0.5em, minimum size=0.6cm, text centered, font=\tiny},
    flecha/.style={->, >=latex', shorten >=1pt, thick},
    etiqueta/.style={text centered, font=\tiny} 
}  
\node[bloque] (bsize) {1};
\node[bloque, right=0cm of bsize] (bcap) {4};
\node[bloque,right=0cm of bcap] (bptr) { };
\node[bloque,right=0.5cm of bptr] (v0) {1.0};
\node[bloquelibre,right=0cm of v0] (v1) {2.0};
\node[bloquelibre,right=0cm of v1] (v2) {0.0};
\node[bloquelibre,right=0cm of v2] (v3) {?};
\draw[flecha] (bptr) -- (v0);
\node[etiqueta, left=0.1cm of bsize] {v1:};
\node[etiqueta, above=0cm of bsize] {tam};
\node[etiqueta, above=0cm of bcap] {cap};
\node[etiqueta, above=0cm of bptr] {vec};
\end{tikzpicture}

\mode<presentation>{\vfill\pause}
\begin{itemize}
  \item \cppid{v.redimensiona(6)}
\end{itemize}
\pause
\begin{tikzpicture}
\tikzset{
    bloque/.style={rectangle,draw=black, top color=white, bottom color=blue!50,
                   very thick, inner sep=0.5em, minimum size=0.6cm, text centered, font=\tiny},
    bloquelibre/.style={rectangle,draw=black, top color=white, bottom color=blue!10,
                   very thick, inner sep=0.5em, minimum size=0.6cm, text centered, font=\tiny},
    flecha/.style={->, >=latex', shorten >=1pt, thick},
    etiqueta/.style={text centered, font=\tiny} 
}  
\node[bloque] (bsize) {6};
\node[bloque, right=0cm of bsize] (bcap) {6};
\node[bloque,right=0cm of bcap] (bptr) { };

\node[bloque,right=0.5cm of bptr] (v0) {1.0};
\node[bloquelibre,right=0cm of v0] (v1) {2.0};
\node[bloquelibre,right=0cm of v1] (v2) {0.0};
\node[bloquelibre,right=0cm of v2] (v3) {?};
\node[etiqueta, right=0.5cm of v3] (delete) {\cppkey{delete[]}};
\draw[flecha] (delete) -- (v3);

\node[bloque,below right=0.5cm and 0.5cm of bptr] (nv0) {1.0};
\node[bloque,right=0cm of nv0] (nv1) {2.0};
\node[bloque,right=0cm of nv1] (nv2) {0.0};
\node[bloque,right=0cm of nv2] (nv3) {0.0};
\node[bloque,right=0cm of nv3] (nv4) {0.0};
\node[bloque,right=0cm of nv4] (nv5) {0.0};

\draw[flecha] (bptr) -- (nv0);

\node[etiqueta, left=0.1cm of bsize] {v1:};
\node[etiqueta, above=0cm of bsize] {tam};
\node[etiqueta, above=0cm of bcap] {cap};
\node[etiqueta, above=0cm of bptr] {vec};
\end{tikzpicture}
\end{frame}

\begin{frame}[t,fragile]{Cambio de tamaño}
\begin{block}{vecnum.cpp}
\begin{lstlisting}
void vecnum::redimensiona(int n) {
  if (n < 0) { n = 0; }
  reserva(n);
  std::fill_n(buffer_.get() + tamanyo_, n - tamanyo_, 0.0);
  tamanyo_ = n;
}
\end{lstlisting}
\end{block}
\end{frame}

\begin{frame}[t]{Agregación de elementos}
\begin{itemize}
  \item \cppid{agrega\_final(x)}: Agrega un elemento al final del vector 
        modificando su \textmark{tamaño}.
    \begin{itemize}
      \item Si no hay espacio disponible se reserva capacidad adicional.
    \end{itemize}
\end{itemize}

\mode<presentation>{\vfill\pause}
\begin{tikzpicture}
\tikzset{
    bloque/.style={rectangle,draw=black, top color=white, bottom color=blue!50,
                   very thick, inner sep=0.5em, minimum size=0.6cm, text centered, font=\tiny},
    bloquelibre/.style={rectangle,draw=black, top color=white, bottom color=blue!10,
                   very thick, inner sep=0.5em, minimum size=0.6cm, text centered, font=\tiny},
    flecha/.style={->, >=latex', shorten >=1pt, thick},
    etiqueta/.style={text centered, font=\tiny} 
}  
\node[bloque] (bsize) {2};
\node[bloque, right=0cm of bsize] (bcap) {4};
\node[bloque,right=0cm of bcap] (bptr) { };
\node[bloque,right=0.5cm of bptr] (v0) {1.0};
\node[bloque,right=0cm of v0] (v1) {2.0};
\node[bloquelibre,right=0cm of v1] (v2) {?};
\node[bloquelibre,right=0cm of v2] (v3) {?};
\draw[flecha] (bptr) -- (v0);
\node[etiqueta, left=0.1cm of bsize] {v1:};
\node[etiqueta, above=0cm of bsize] {tam};
\node[etiqueta, above=0cm of bcap] {cap};
\node[etiqueta, above=0cm of bptr] {vec};
\end{tikzpicture}

\mode<presentation>{\vfill\pause}
\begin{itemize}
  \item \cppid{agrega\_final(1.5)}
\end{itemize}
\pause
\begin{tikzpicture}
\tikzset{
    bloque/.style={rectangle,draw=black, top color=white, bottom color=blue!50,
                   very thick, inner sep=0.5em, minimum size=0.6cm, text centered, font=\tiny},
    bloquelibre/.style={rectangle,draw=black, top color=white, bottom color=blue!10,
                   very thick, inner sep=0.5em, minimum size=0.6cm, text centered, font=\tiny},
    flecha/.style={->, >=latex', shorten >=1pt, thick},
    etiqueta/.style={text centered, font=\tiny} 
}  
\node[bloque] (bsize) {3};
\node[bloque, right=0cm of bsize] (bcap) {4};
\node[bloque,right=0cm of bcap] (bptr) { };
\node[bloque,right=0.5cm of bptr] (v0) {1.0};
\node[bloque,right=0cm of v0] (v1) {2.0};
\node[bloque,right=0cm of v1] (v2) {1.5};
\node[bloquelibre,right=0cm of v2] (v3) {?};
\draw[flecha] (bptr) -- (v0);
\node[etiqueta, left=0.1cm of bsize] {v1:};
\node[etiqueta, above=0cm of bsize] {tam};
\node[etiqueta, above=0cm of bcap] {cap};
\node[etiqueta, above=0cm of bptr] {vec};
\end{tikzpicture}

\mode<presentation>{\vfill\pause}
\begin{itemize}
  \item \cppid{agrega\_final(2.5)}
\end{itemize}
\pause
\begin{tikzpicture}
\tikzset{
    bloque/.style={rectangle,draw=black, top color=white, bottom color=blue!50,
                   very thick, inner sep=0.5em, minimum size=0.6cm, text centered, font=\tiny},
    bloquelibre/.style={rectangle,draw=black, top color=white, bottom color=blue!10,
                   very thick, inner sep=0.5em, minimum size=0.6cm, text centered, font=\tiny},
    flecha/.style={->, >=latex', shorten >=1pt, thick},
    etiqueta/.style={text centered, font=\tiny} 
}  
\node[bloque] (bsize) {4};
\node[bloque, right=0cm of bsize] (bcap) {4};
\node[bloque,right=0cm of bcap] (bptr) { };
\node[bloque,right=0.5cm of bptr] (v0) {1.0};
\node[bloque,right=0cm of v0] (v1) {2.0};
\node[bloque,right=0cm of v1] (v2) {1.5};
\node[bloque,right=0cm of v2] (v3) {2.5};
\draw[flecha] (bptr) -- (v0);
\node[etiqueta, left=0.1cm of bsize] {v1:};
\node[etiqueta, above=0cm of bsize] {tam};
\node[etiqueta, above=0cm of bcap] {cap};
\node[etiqueta, above=0cm of bptr] {vec};
\end{tikzpicture}
\end{frame}

\begin{frame}[t]{Agregación de elementos}
\begin{tikzpicture}
\tikzset{
    bloque/.style={rectangle,draw=black, top color=white, bottom color=blue!50,
                   very thick, inner sep=0.5em, minimum size=0.6cm, text centered, font=\tiny},
    bloquelibre/.style={rectangle,draw=black, top color=white, bottom color=blue!10,
                   very thick, inner sep=0.5em, minimum size=0.6cm, text centered, font=\tiny},
    flecha/.style={->, >=latex', shorten >=1pt, thick},
    etiqueta/.style={text centered, font=\tiny} 
}  
\node[bloque] (bsize) {4};
\node[bloque, right=0cm of bsize] (bcap) {4};
\node[bloque,right=0cm of bcap] (bptr) { };
\node[bloque,right=0.5cm of bptr] (v0) {1.0};
\node[bloque,right=0cm of v0] (v1) {2.0};
\node[bloque,right=0cm of v1] (v2) {1.5};
\node[bloque,right=0cm of v2] (v3) {2.5};
\draw[flecha] (bptr) -- (v0);
\node[etiqueta, left=0.1cm of bsize] {v1:};
\node[etiqueta, above=0cm of bsize] {tam};
\node[etiqueta, above=0cm of bcap] {cap};
\node[etiqueta, above=0cm of bptr] {vec};
\end{tikzpicture}

\mode<presentation>{\vfill\pause}
\begin{itemize}
  \item \cppid{agrega\_final(3.5)}
\end{itemize}

\begin{tikzpicture}
\tikzset{
    bloque/.style={rectangle,draw=black, top color=white, bottom color=blue!50,
                   very thick, inner sep=0.5em, minimum size=0.6cm, text centered, font=\tiny},
    bloquelibre/.style={rectangle,draw=black, top color=white, bottom color=blue!10,
                   very thick, inner sep=0.5em, minimum size=0.6cm, text centered, font=\tiny},
    flecha/.style={->, >=latex', shorten >=1pt, thick},
    etiqueta/.style={text centered, font=\tiny} 
}  
\node[bloque] (bsize) {5};
\node[bloque, right=0cm of bsize] (bcap) {8};
\node[bloque,right=0cm of bcap] (bptr) { };

\node[bloque,right=0.5cm of bptr] (v0) {1.0};
\node[bloque,right=0cm of v0] (v1) {2.0};
\node[bloque,right=0cm of v1] (v2) {1.5};
\node[bloque,right=0cm of v2] (v3) {2.5};
\node[etiqueta, right=0.5cm of v3] (delete) {\cppkey{delete[]}};
\draw[flecha] (delete) -- (v3);

\node[bloque,below right=0.5cm and 0.5cm of bptr] (nv0) {1.0};
\node[bloque,right=0cm of nv0] (nv1) {2.0};
\node[bloque,right=0cm of nv1] (nv2) {1.5};
\node[bloque,right=0cm of nv2] (nv3) {2.5};
\node[bloque,right=0cm of nv3] (nv4) {3.5};
\node[bloquelibre,right=0cm of nv4] (nv5) {?};
\node[bloquelibre,right=0cm of nv5] (nv6) {?};
\node[bloquelibre,right=0cm of nv6] (nv7) {?};

\draw[flecha] (bptr) -- (nv0);

\node[etiqueta, left=0.1cm of bsize] {v1:};
\node[etiqueta, above=0cm of bsize] {tam};
\node[etiqueta, above=0cm of bcap] {cap};
\node[etiqueta, above=0cm of bptr] {vec};
\end{tikzpicture}
\end{frame}

\begin{frame}[t,fragile]{Agregación de elementos}
\begin{block}{vencum.cpp}
\begin{lstlisting}
void vecnum::agrega_final(double x) {
  constexpr int capacidad_inicial = 8;
  constexpr double factor = 1.8;

  if (capacidad_ == 0) {
    reserva(capacidad_inicial);
  } else {
    reserva(static_cast<int>(factor * capacidad_));
  }
  buffer_[tamanyo_++] = x;
}
\end{lstlisting}
\end{block}
\end{frame}
