\section{Una lista enlazada}

\subsection{Diseño}

\begin{frame}[t]{Estructuras de datos enlazadas}
\begin{itemize}
  \item \textmark{Objetivo}: Representar una lista dinámica de valores
        manteniendo punteros al siguiente elemento y al anterior.

\mode<presentation>{\vfill\pause}
\begin{tikzpicture}
\node[bloquelibre] (nullizq) {\cppkey{nullptr}};

\node[bloquelibre, right=0.5cm of nullizq] (ant1) { };
\node[bloque, right=0cm of ant1] (val1) {3.0};
\node[bloquelibre, right=0cm of val1] (sig1) { };

\node[bloquelibre, right=0.5cm of sig1] (ant2) { };
\node[bloque, right=0cm of ant2] (val2) {4.0};
\node[bloquelibre, right=0cm of val2] (sig2) { };

\node[bloquelibre, right=0.5cm of sig2] (ant3) { };
\node[bloque, right=0cm of ant3] (val3) {5.0};
\node[bloquelibre, right=0cm of val3] (sig3) { };

\node[bloquelibre, right=0.5cm of sig3] (nullder) {\cppkey{nullptr}};
\draw[flecha] (ant1) -- (nullizq);
\draw[flecha] (sig1) to [bend right=30] (ant2);
\draw[flecha] (sig2) to [bend right=30] (ant3);
\draw[flecha] (ant2) to [bend right=30] (sig1);
\draw[flecha] (ant3) to [bend right=30] (sig2);
\draw[flecha] (sig3) -- (nullder);
\end{tikzpicture}

  \mode<presentation>{\vfill\pause}
  \item \textmark{Efectos}:
    \begin{itemize}
      \item No hace falta reservar capacidad extra para crecimiento.
      \item No hay que copiar elementos al reasignar memoria.
      \item Solamente se consume la memoria que se necesita.
      \item Sobrecoste: dos punteros por nodo.
      \item Puede tener efectos adversos con caché de procesador.
    \end{itemize}
\end{itemize}
\end{frame}

\begin{frame}[t]{Diseño}

\begin{columns}[T]

\column{.6\textwidth}
\begin{itemize}
  \item Estrategia de \textmark{nodo centinela}:
    \begin{itemize}
      \item Representar la lista mediante dos punteros.
        \begin{itemize}
          \item \cppid{inicio}: Primer nodo de la lista.
          \item \cppid{fin}: Último nodo de la lista.
        \end{itemize}
      \item Se usa un nodo adicional al final de la lista.
    \end{itemize}
\end{itemize}

\column{.1\textwidth}
\column{.4\textwidth}
\pause
\begin{itemize}
  \item Lista vacía:
\end{itemize}
\mode<presentation>{\vspace{1em}}
\begin{tikzpicture}
\node[bloquelibre] (ant1) {\cppkey{nullptr}};
\node[bloque, right=0cm of ant1] (valor1) {\cppid{?}};
\node[bloquelibre,right=0cm of valor1] (sig1) {\cppkey{nullptr}};

\node[bloquelibre,below=0.5cm of ant1] (inicio) { };
\node[etiqueta,below=0cm of inicio] {\cppid{inicio}};
\draw[flecha] (inicio) -- (valor1);
\node[bloquelibre,below=0.5cm of sig1] (fin) { };
\draw[flecha] (fin) -- (valor1);
\node[etiqueta,below=0cm of fin] {\cppid{fin}};
\end{tikzpicture}

\end{columns}

  \mode<presentation>{\vfill\pause}
  \begin{itemize}
    \item Lista \cppid{\{1,2,3\}}:
  \end{itemize}

\mode<presentation>{\vspace{1em}}
\begin{tikzpicture}
\node[bloquelibre] (ant1) {\cppkey{nullptr}};
\node[bloque, right=0cm of ant1] (valor1) {\cppid{1}};
\node[bloquelibre,right=0cm of valor1] (sig1) { };

\node[bloquelibre,right=0.5cm of sig1] (ant2) {};
\node[bloque, right=0cm of ant2] (valor2) {\cppid{2}};
\node[bloquelibre,right=0cm of valor2] (sig2) {};

\draw[flecha] (sig1) to[bend right=30] (ant2);
\draw[flecha] (ant2) to[bend right=30] (sig1);

\node[bloquelibre,right=0.5cm of sig2] (ant3) { };
\node[bloque, right=0cm of ant3] (valor3) {\cppid{3}};
\node[bloquelibre,right=0cm of valor3] (sig3) { };

\draw[flecha] (sig2) to[bend right=30] (ant3);
\draw[flecha] (ant3) to[bend right=30] (sig2);

\node[bloquelibre,right=0.5cm of sig3] (ant4) { };
\node[bloque, right=0cm of ant4] (valor4) {\cppid{?}};
\node[bloquelibre,right=0cm of valor4] (sig4) {\cppkey{nullptr}};

\draw[flecha] (sig3) to[bend right=30] (ant4);
\draw[flecha] (ant4) to[bend right=30] (sig3);

\node[bloquelibre,below=0.5cm of ant1] (inicio) { };
\node[etiqueta,below=0cm of inicio] {\cppid{inicio}};
\draw[flecha] (inicio) -- (valor1);
\node[bloquelibre,below=0.5cm of sig4] (fin) { };
\draw[flecha] (fin) -- (valor4);
\node[etiqueta,below=0cm of fin] {\cppid{fin}};
\end{tikzpicture}
\end{frame}

\subsection{Representación y operaciones}

\begin{frame}[t,fragile]{Representación interna}
\begin{block}{listnum.hpp}
\begin{lstlisting}
class listnum {
  //...
private:
  struct node {
    explicit node(double x) : valor_{x} {}
    double valor_;
    std::shared_ptr<node> sig_;
    std::weak_ptr<node> ant_;
  };

  std::shared_ptr<node> inicio_;
  std::shared_ptr<node> fin_;
  //...
};
\end{lstlisting}
\end{block}
\end{frame}

\begin{frame}[t,fragile]{Constructores}
\begin{block}{listnum.hpp}
\begin{lstlisting}
class listnum {
public:
  listnum();
  explicit listnum(int n);
  listnum(std::initializer_list<double> l);

  listnum(const listnum & l);
  listnum & operator=(const listnum & l);

  listnum(listnum &&) = default;
  listnum & operator=(listnum &&) = default;

  ~listnum() = default;
  //...
}
\end{lstlisting}
\end{block}
\end{frame}


\begin{frame}[t,fragile]{Operaciones}
\begin{block}{listnum.hpp}
\begin{lstlisting}
class listnum {
public:
  //...
  [[nodiscard]] int tamanyo() const noexcept;

  void agrega_final(double x);
  //...
}
\end{lstlisting}
\end{block}
\end{frame}

\subsection{Agregación de elementos}

\begin{frame}[t,fragile]{Agregando elementos en una lista vacía}
\begin{itemize}
  \item Lista vacía:
\end{itemize}
\mode<presentation>{\vspace{1em}}
\begin{tikzpicture}


\node[bloquelibre,minimum width=2.6em] (ant1) {\cppkey{nullptr}};
\node[bloque, right=0cm of ant1] (valor1) {\cppid{?}};
\node[bloquelibre,right=0cm of valor1] (sig1) {\cppkey{nullptr}};

\node[bloquelibre,below=0.5cm of ant1] (inicio) { };
\node[etiqueta,below=0cm of inicio] {\cppid{inicio}};
\draw[flecha] (inicio) -- (valor1);
\node[bloquelibre,below=0.5cm of sig1] (fin) { };
\draw[flecha] (fin) -- (valor1);
\node[etiqueta,below=0cm of fin] {\cppid{fin}};

\pause
\node[bloquelibre,left=2cm of ant1] (sig0) {\cppkey{nullptr}};
\node[bloque, left=0cm of sig0] (valor0) {\cppid{x}};
\node[bloquelibre,left=0cm of valor0] (ant0) {\cppkey{nullptr}};

\pause
\node[bloquelibre, right=0cm of valor0,minimum width=2.6em] (sig0b) { };
\draw[flecha,color=red] (sig0) to[bend right=30] (ant1);
\pause
\node[bloquelibre, left=0cm of valor1,minimum width=2.6em] (ant1b) { };
\draw[flecha,color=red] (ant1) to[bend right=30] (sig0);
\pause
\draw[flecha,color=red] (inicio.west) to[bend left=60] (valor0.south);
\draw[flecha,color=white] (inicio) -- (valor1);
\end{tikzpicture}
\end{frame}


\begin{frame}[t,fragile]{Agregando elementos}
\begin{itemize}
  \item Lista \cppid{\{1,2,3\}}
\end{itemize}
\begin{tikzpicture}
\node[bloquelibre] (ant1) {\cppkey{nullptr}};
\node[bloque, right=0cm of ant1] (valor1) {\cppid{1}};
\node[bloquelibre,right=0cm of valor1] (sig1) { };

\node[bloquelibre,right=0.5cm of sig1] (ant2) {};
\node[bloque, right=0cm of ant2] (valor2) {\cppid{2}};
\node[bloquelibre,right=0cm of valor2] (sig2) {};

\draw[flecha] (sig1) to[bend right=30] (ant2);
\draw[flecha] (ant2) to[bend right=30] (sig1);

\node[bloquelibre,right=0.5cm of sig2] (ant3) { };
\node[bloque, right=0cm of ant3] (valor3) {\cppid{3}};
\node[bloquelibre,right=0cm of valor3] (sig3) { };

\draw[flecha] (sig2) to[bend right=30] (ant3);
\draw[flecha] (ant3) to[bend right=30] (sig2);

\node[bloquelibre,right=0.5cm of sig3] (ant4) { };
\node[bloque, right=0cm of ant4] (valor4) {\cppid{?}};
\node[bloquelibre,right=0cm of valor4] (sig4) {\cppkey{nullptr}};

\draw[flecha] (sig3) to[bend right=30] (ant4);
\draw[flecha] (ant4) to[bend right=30] (sig3);

\node[bloquelibre,below=0.5cm of ant1] (inicio) { };
\node[etiqueta,below=0cm of inicio] {\cppid{inicio}};
\draw[flecha] (inicio) -- (valor1);
\node[bloquelibre,below=0.5cm of sig4] (fin) { };
\draw[flecha] (fin) -- (valor4);
\node[etiqueta,below=0cm of fin] {\cppid{fin}};

\pause
\node[bloque,above right=1cm and 0cm of sig3] (valor0) {\cppid{x}};
\node[bloquelibre,left=0cm of valor0,minimum width=2.6em] (ant0) {\cppkey{nullptr}};
\node[bloquelibre,right=0cm of valor0,minimum width=2.6em] (sig0) {\cppkey{nullptr}};

\pause
\node[bloquelibre,left=0cm of valor0,minimum width=2.6em] (ant0) {};
\node[bloquelibre,right=0cm of valor0,minimum width=2.6em] (sig0) {};
\draw[flecha,color=red] (ant0) -- (valor3);
\draw[flecha,color=red] (sig0) -- (valor4);

\pause
\draw[flecha,color=white] (sig3) to[bend right=30] (ant4);
\draw[flecha,color=white] (ant4) to[bend right=30] (sig3);
\draw[flecha,color=red] (sig3) -- (valor0);
\draw[flecha,color=red] (ant4) -- (valor0);
\end{tikzpicture}
\end{frame}

\begin{frame}[t,fragile]{Agregando elementos}
\begin{block}{listnum.cpp}
\begin{lstlisting}
void listnum::agrega_final(double x) {
  auto aux = std::make_shared<node>(x);
  aux->ant_ = fin_->ant_;
  aux->sig_ = fin_;
  if (inicio_ == fin_) {
    inicio_ = aux;
  } else {
    fin_->ant_.lock()->sig_ = aux;
  }
  fin_->ant_ = aux;
}
\end{lstlisting}
\end{block}
\end{frame}

\subsection{Construcción y copia}

\begin{frame}[t,fragile]{Construcción}
\begin{block}{listnum.cpp}
\begin{lstlisting}
listnum::listnum() 
    : inicio_{std::make_shared<node>(0)}, 
      fin_{inicio_} {}

listnum::listnum(int n) 
    : listnum() // Invocación a constructor por defecto
{
  for (int i = 0; i < n; ++i) { agrega_final(0.0); }
}

listnum::listnum(std::initializer_list<double> l) 
    : listnum() // Invocación a constructor vacío
{
  for (auto x : l) { agrega_final(x); }
}
\end{lstlisting}
\end{block}
\end{frame}

\begin{frame}[t,fragile]{Constructor de copia}
\begin{block}{listnum.cpp}
\begin{lstlisting}
listnum::listnum(const listnum & l) 
    : listnum() 
{
  std::shared_ptr<listnum::node> aux = l.inicio_;
  while (aux != l.fin_) {
    agrega_final(aux->valor_);
    aux = aux->sig_;
  }
}
\end{lstlisting}
\end{block}
\end{frame}

\begin{frame}[t,fragile]{Operador de asignación de copia}
\begin{block}{listnum.cpp}
\begin{lstlisting}
listnum & listnum::operator=(const listnum & l) {
  if (this == &l) { return *this; }
  std::shared_ptr<listnum::node> aux = l.inicio_;
  while (aux != l.fin_) {
    agrega_final(aux->valor_);
    aux = aux->sig_;
  }
  return *this;
}
\end{lstlisting}
\end{block}
\end{frame}

\subsection{Tamaño}

\begin{frame}[t,fragile]{Cálculo del tamaño}
\begin{block}{listnum.cpp}
\begin{lstlisting}
int listnum::tamanyo() const noexcept {
  std::weak_ptr<listnum::node> aux = inicio_;
  int n = 0;
  while (aux.lock() != fin_) {
    n++;
    aux = aux.lock()->sig_;
  }
  return n;
}
\end{lstlisting}
\end{block}
\end{frame}

\subsection{Iteración}

\begin{frame}[t,fragile]{Iteradores}
\begin{itemize}
  \item Un \textgood{iterador} representa un mecanismo de acceso a una posición
        en el contenedor.

  \mode<presentation>{\vfill\pause}
  \item \textmark{Operaciones}:
    \begin{itemize}
      \item \textgood{Desreferenciar}: Acceder al elemento apuntado.
\begin{lstlisting}
x = *it;
*it = x;
\end{lstlisting}

      \pause
      \item \textgood{Avanzar}: Avanza a la siguiente posición.
\begin{lstlisting}
++it;
\end{lstlisting}

      \pause
      \item \textgood{Comparar}: Compara si dos iteradores apuntan a la misma posición
\begin{lstlisting}
if (it1 == it2) //...
if (it1 != it2) //...
\end{lstlisting}
    \end{itemize}
\end{itemize}
\end{frame}

\begin{frame}[t,fragile]{Iterador}
\begin{block}{listnum.cpp}
\begin{lstlisting}
class listnum {
  public:
    //...
    class iterador {
    public:
      explicit iterador(const std::shared_ptr<node> & p) : actual_{p} {}
      //...
    private:
      std::weak_ptr<node> actual_;
    };
    //...
};
\end{lstlisting}
\end{block}
\begin{itemize}
  \item Un iterador \textmark{encapsula} un \cppid{weak\_ptr} con una 
        \textbad{interfaz restringida}.
\end{itemize}
\end{frame}


\begin{frame}[t,fragile]{Iterador}
\mode<presentation>{\vspace{-.75em}}
\begin{block}{listnum.cpp}
\mode<presentation>{\vspace{-.25em}}
\begin{lstlisting}
class listnum {
  public:
    class iterador {
    public:
      //...
      double & operator*() { return actual_.lock()->valor_; }
      iterador operator++() {
        actual_ = actual_.lock()->sig_;
        return *this;
      }
      bool operator==(const iterador & it) const {
        return actual_.lock() == it.actual_.lock();
      }
      bool operator!=(const iterador & it) const {
        return actual_.lock() != it.actual_.lock();
      }
      //...
    };
};
\end{lstlisting}
\end{block}
\end{frame}

\mode<article>{
\subsection{Listados completos de \textbf{listnum}}

\begin{frame}[t]{listnum.hpp}
\lstinputlisting{ejemplos/05-memmgmt/listnum/listnum.hpp}
\end{frame}

\begin{frame}[t]{listnum.cpp}
\lstinputlisting{ejemplos/05-memmgmt/listnum/listnum.cpp}
\end{frame}

}
