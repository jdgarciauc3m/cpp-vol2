\section{Semantica de propiedad y \textbf{unique\_ptr}}

\begin{frame}[t]{Semántica de propiedad estricta}
\begin{itemize}
  \item El tipo \cppid{unique\_ptr} ofrece \textmark{semántica de propiedad estricta}.
    \begin{itemize}
      \item \textbad{Nunca} hay dos \cppid{unique\_ptr} apuntando al mismo 
            \textgood{bloque de memoria dinámica}.
    \end{itemize}

  \mode<presentation>{\vfill\pause}
  \item \textmark{Ciclo de vida}:
    \begin{itemize}
      \item El bloque de memoria se \textgood{adquiere} con el \textemph{constructor}.
      \item El bloque de memoria se \textgood{libera} con el \textemph{destructor}.
    \end{itemize}

  \mode<presentation>{\vfill\pause}
  \item \textmark{Efecto}:
    \begin{itemize}
      \item Un \cppid{unique\_ptr} apunta a un \textgood{bloque de memoria dinámica}
            o al puntero nulo \cppkey{nullptr}.
    \end{itemize}

  \mode<presentation>{\vfill\pause}
  \item \textmark{Aproximación}:
    \begin{itemize}
      \item Un \cppid{unique\_ptr} \textbad{no se puede copiar}.
      \item Un \cppid{unique\_ptr} \textemph{se puede mover}.
    \end{itemize}
\end{itemize}
\end{frame}

\begin{frame}[t,fragile]{Copia}
\begin{itemize}
  \item El \textmark{constructor de copia} de \cppid{unique\_ptr} 
        está \textbad{eliminado}.
    \begin{itemize}
      \item Pero se puede construir a partir de un \textmark{puntero primitivo}
            que apunte a memoria dinámica adquiriendo su propiedad.
      \item El constructor es \textemph{explícito}.
    \end{itemize}
\begin{lstlisting}
std::unique_ptr<double> p{new double{42}}; // OK. Iniciación directa
std::unique_ptr<double> q = new double{42}; // Error. Constructor explícito.
\end{lstlisting}

  \mode<presentation>{\vfill\pause}
  \item El \textmark{operador de asignación de copia} de \cppid{unique\_ptr}
        está \textbad{eliminado}.
    \begin{itemize}
      \item No se puede asignar un \textmark{puntero primitivo}
            a un \cppid{unique\_ptr}.
      \item Pero se puede conseguir este efecto mediante la operación \cppid{reset(()}.
    \end{itemize}
\begin{lstlisting}
std::unique_ptr<double> p;
p = new double{42}; // Error
p.reset(new double{42}); // OK
\end{lstlisting}
\end{itemize}
\end{frame}

\begin{frame}[t,fragile]{Asignación y \textbf{nullptr}}
\begin{itemize}
  \item En general, no se puede asignar un \textmark{puntero primitivo} a un
        \cppid{unique\_ptr}.
\begin{lstlisting}
std::unique_ptr<double> p{new double{42}};
p = new double{99}; // Error
\end{lstlisting}

  \mode<presentation>{\vfill\pause}
  \item Pero sí se puede asignar a un \cppid{unique\_ptr} el valor \cppkey{nullptr}.
\begin{lstlisting}
std::unique_ptr<double> p{new double{42}};
//...
p = nullptr; // Libera la memoria poseida por p
\end{lstlisting}

  \mode<presentation>{\vfill\pause}
  \item O se puede usar \cppkey{nullptr} como argumento para \cppid{reset()}.
\begin{lstlisting}
std::unique_ptr<double> p{new double{42}};
//...
p.reset(nullptr); // Libera la memoria poseida por p
p.reset(); // Equivale a p.resent(nullptr)
\end{lstlisting}
\end{itemize}
\end{frame}

\begin{frame}[t,fragile]{Movimiento}
\begin{itemize}
  \item El \textgood{constructor de movimiento} de \cppid{unique\_ptr} permite
        la \textmark{transferencia de la propiedad} de un bloque de memoria.
    \begin{itemize}
      \item No es \textemph{explícito}.
    \end{itemize}
\begin{lstlisting}
auto p = std::make_unique<double>(42); // OK. Iniciación de copia.
auto q{std::make_unique<double>(33)}; // OK. Iniciación directa.
auto r = p; // Error
auto s = std::move(p); // *s==42, p==nullptr
\end{lstlisting}

  \mode<presentation>{\vfill\pause}
  \item El \textgood{operador de asignación de movimiento} de \cppid{unique\_ptr}
        permite la \textmark{transferencia de la propiedad} de un bloque de memoria.
    \begin{itemize}
      \item Libera el bloque de memoria previo.
    \end{itemize}
\begin{lstlisting}
auto p = std::make_unique<double>(42);
p = std::make_unique<double>(33); // Movimiento. Libera bloque previo.
auto q = std::make_unique<double>(99);
q = p; // Error
q = std::move(p); // Movimiento
\end{lstlisting}
\end{itemize}
\end{frame}

\begin{frame}[t,fragile]{Transferencia de propiedad y funciones}
\begin{itemize}
  \item Funciones \textmark{sumidero}:
\begin{lstlisting}
void f(std::unique_ptr<double> p) { /* ... */ }
//...
auto p = std::make_shared<double>{42};
//...
f(p); // Error. No se puede copiar
f(std::move(p)); // OK. Mueve p a función f()
// p==nullptr
\end{lstlisting}

  \mode<presentation>{\vfill\pause}
  \item Funciones \textmark{fuente}:
\begin{lstlisting}
std::unique_ptr<double> make_array(int n) {
  auto r = std::make_unique<double[]>(n);
  for (int i=0; i<n; ++i) { r[i]=std::sqrt(i); };
  return r; 
}
//...
auto v = make_array(10); // Movimiento
\end{lstlisting}
\end{itemize}
\end{frame}
