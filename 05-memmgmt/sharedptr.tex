\section{Punteros de propiedad compartida}

\subsection{Introducción}

\begin{frame}[t,fragile]{Punteros compartidos}
\begin{itemize}
  \item Un \textgood{puntero compartido} ofrece el concepto de
        \textmark{propiedad compartida}.
    \begin{itemize}
      \pause
      \item Varios \textgood{punteros compartidos} pueden apuntar al mismo
            objeto en memoria dinámica.

      \pause
      \item Un \textgood{puntero compartido} tiene asociado un 
            \textmark{contador de referencias}.

      \pause
      \item Cuando el último \textgood{puntero compartido} se destruye,
            se libera el objeto en memoria dinámica.
    \end{itemize}

  \mode<presentation>{\vfill\pause}
  \item El tipo \cppid{shared\_ptr} ofrece semántica de 
        \textmark{propiedad compartida}.
\begin{lstlisting}[escapechar=@]
void f() {
  std::shared_ptr<std::string> p {new std::string>{"Daniel"}}; // ref==1@\pause@
  auto q = p; // ref==2@\pause@
  std::cout << *q << " -> " << q->size() << "\n"; // Daniel 6
  p = nullptr; // ref==1, q->bloque@\pause@
} // Destrucción p -> ref==0 -> Liberación memoria
\end{lstlisting}

\end{itemize}
\end{frame}

\begin{frame}[t,fragile]{Contador de referencias}

\begin{columns}[T]

\column{.5\textwidth}

\begin{lstlisting}
p = new std::string{"Daniel"};
\end{lstlisting}

\mode<presentation>{\vspace{-0.25cm}}
\begin{tikzpicture}
\node[bloque] (pobj) { };
\node[etiqueta, left=0cm of pobj] {\cppid{p}};

\node[bloque, right=0.5cm of pobj] (pref) {1};
\node[etiqueta, above=0cm of pref] {\cppid{ref}};
\node[bloque, right=0cm of pref] (pptr) { };
\node[etiqueta, above=0cm of pptr] {\cppid{ptr}};
\draw[flecha] (pobj) -- (pref);

\node[bloquelibre, right=0.5cm of pptr] (pstr) {\cppstr{"Daniel"}};
\draw[flecha] (pptr) -- (pstr);
\end{tikzpicture}

\mode<presentation>{\vspace{3em}\vfill\pause}
\begin{lstlisting}
auto q = p;
\end{lstlisting}

\mode<presentation>{\vspace{-0.25cm}}
\begin{tikzpicture}
\node[bloque] (pobj) { };
\node[etiqueta, left=0cm of pobj] {\cppid{p}};

\node[bloque, right=0.5cm of pobj] (pref) {2};
\node[etiqueta, above=0cm of pref] {\cppid{ref}};
\node[bloque, right=0cm of pref] (pptr) { };
\node[etiqueta, above=0cm of pptr] {\cppid{ptr}};
\draw[flecha] (pobj) -- (pref);

\node[bloquelibre, right=0.5cm of pptr] (pstr) {\cppstr{"Daniel"}};
\draw[flecha] (pptr) -- (pstr);

\node[bloque, below=0.25cm of pobj] (qobj) { };
\node[etiqueta, left=0cm of qobj] {\cppid{q}};
\draw[flecha] (qobj) -- (pref);
\end{tikzpicture}

\column{.5\textwidth}

\begin{lstlisting}
p=nullptr;
\end{lstlisting}

\mode<presentation>{\vspace{-0.25cm}}
\begin{tikzpicture}
\node[bloque] (pobj) {\cppkey{nullptr}};
\node[etiqueta, left=0cm of pobj] {\cppid{p}};

\node[bloque, right=0.5cm of pobj] (pref) {0};
\node[etiqueta, above=0cm of pref] {\cppid{ref}};
\node[bloque, right=0cm of pref] (pptr) { };
\node[etiqueta, above=0cm of pptr] {\cppid{ptr}};

\node[bloquelibre, right=0.5cm of pptr] (pstr) {\cppstr{"Daniel"}};
\draw[flecha] (pptr) -- (pstr);

\node[bloque, below=0.25cm of pobj] (qobj) { };
\node[etiqueta, left=0cm of qobj] {\cppid{q}};
\draw[flecha] (qobj) -- (pref);
\end{tikzpicture}


\mode<presentation>{\vspace{1em}\vfill\pause}
\mode<presentation>{\vfill\pause}
\begin{lstlisting}
} // Destrucción de q
\end{lstlisting}

\mode<presentation>{\vspace{-0.25cm}}
\begin{tikzpicture}
\node[bloque] (pobj) {\cppkey{nullptr}};
\node[etiqueta, left=0cm of pobj] {\cppid{p}};

\node[bloque, right=0.5cm of pobj] (pref) {1};
\node[etiqueta, above=0cm of pref] {\cppid{ref}};
\node[bloque, right=0cm of pref] (pptr) { };
\node[etiqueta, above=0cm of pptr] {\cppid{ptr}};

\node[bloquelibre, right=0.5cm of pptr] (pstr) {\cppstr{"Daniel"}};
\draw[flecha] (pptr) -- (pstr);
\node[etiqueta, above=0.5cm of pstr] (delete) {\cppkey{delete[]}};
\draw[flecha] (delete) -- (pstr);

\node[bloque, below=0.25cm of pobj] (qobj) { };
\node[etiqueta, left=0cm of qobj] {\cppid{q}};
\end{tikzpicture}

\end{columns}
\end{frame}

\subsection{Construcción, copia y movimiento}

\begin{frame}[t,fragile]{Construcción}
\begin{itemize}
  \item Se puede construir a partir de un \textmark{puntero primitivo}.
    \begin{itemize}
      \item El constructor es \textemph{explícito}.
    \end{itemize}
\begin{lstlisting}
std::shared_ptr<double> p{new double{42.0}}; // OK. Iniciación directa
std::shared_ptr<double> q = new double{1.5}; // Error. Iniciación de copia
\end{lstlisting}

  \mode<presentation>{\vfill\pause}
  \item La construcción por defecto inicia al \textmark{puntero nulo}.
\begin{lstlisting}
std::shared_ptr<double> p; // p==nullptr
std::shared_ptr<double> q{nullptr}; // OK. Iniciación directa
std::shared_ptr<double> r = nullptr; // OK. Iniciación de copia
\end{lstlisting}

\end{itemize}
\end{frame}

\begin{frame}[t,fragile]{Semántica de copia}
\begin{itemize}
  \item Un \cppid{shared\_ptr} soporta \textmark{construcción de copia}.
    \begin{itemize}
      \item \textgood{Incrementa} el \textmark{contador de referencias}.
      \item Ambos punteros \textgood{apuntan} al \textmark{mismo bloque} de memoria dinámica.
    \end{itemize}
\begin{lstlisting}
std::shared_ptr<double> p {new double{42.0}};
auto q = p;
\end{lstlisting}

  \mode<presentation>{\vfill\pause}
  \item Un \cppid{shared\_ptr} soporta \textmark{asignación de copia}.
    \begin{itemize}
      \item \textgood{Decrementa} el \textmark{contador de referencias} del bloque anterior.
      \item \textgood{Libera} el \textmark{bloque de memoria} anterior 
            si se llegó a \cppid{0} referencias.
      \item \textgood{Incrementa} el \textmark{contador de referencias}.
      \item Ambos punteros \textgood{apuntan} al \textmark{mismo bloque} de memoria dinámica.
    \end{itemize}
\begin{lstlisting}
std::shared_ptr<double> p {new double{42.0}};
std::shared_ptr<double> q {new double{99.0}};
p = q;
\end{lstlisting}

\end{itemize}
\end{frame}

\begin{frame}[t,fragile]{Construcción de movimiento}
\begin{itemize}
  \item Un \cppid{shared\_ptr} soporta \textmark{construcción de movimiento}.
    \begin{itemize}
      \item Se transfiere la propiedad del bloque de memoria.
    \end{itemize}
\begin{lstlisting}
std::shared_ptr<double> p {new double{42.0}};
std::shared_ptr<double> q = std::move(p);
\end{lstlisting}

  \mode<presentation>{\vfill\pause}
  \item Un \cppid{shared\_ptr} también se puede \textmark{construir por movimiento}
        a partir de un \cppid{unique\_ptr}.
\begin{lstlisting}[escapechar=@]
auto p = std::make_unique<double>{42.0};
std::shared_ptr<double> q = std::move(p);
@\pause@
std::unique_ptr<cuenta> crea_cuenta();
std::unique_ptr<cuenta> u = crea_cuenta();
std::shared_ptr<cuenta> s = crea_cuenta();
\end{lstlisting}

\end{itemize}
\end{frame}

\begin{frame}[t,fragile]{Funciones auxiliares de creación}
  \begin{itemize}
    \item Funciones equivalentes a 
          \cppid{make\_unique()} y 
          \cppid{make\_unique\_for\_overwrite()}.
      \begin{itemize}
        \item \textgood{Muy útiles} en combinación con semántica de movimiento.
      \end{itemize}

    \mode<presentation>{\vfill\pause}
    \item \cppid{make\_shared()}: Iniciación a valor.
\begin{lstlisting}[escapechar=@]
auto p = std::make_shared<double>(42.0); // *p==42@\pause@
auto q = std::make_shared<punto>(3.5, 4.2); // q->x==3.5, q.y==4.2@\pause@
auto r = std::make_shared<std::string>(); // *r == std::string{}@\pause@
auto v = std::make_shared<int[]>(100); // v[i]== 0@\pause@
auto w = std::make_shared<int[]>(100, 42); // w[i]==42@\pause@
auto x = std::make_shared<double[20]>(); // x[i]==0.0@\pause@
auto y = std::make_shared<double[20]>(1.5); // y[i]==1.5
\end{lstlisting}

    \mode<presentation>{\vfill\pause}
    \item \cppid{make\_shared\_for\_overwrite()}: Iniciación por defecto.
\begin{lstlisting}[escapechar=@]
auto p = std::make_shared_for_overwrite<double>();@\pause@
auto p = std::make_shared_for_overwrite<int[]>(100);
\end{lstlisting}
\textbad{Nota}: No soportado todavía en algunos compiladores.
  \end{itemize}
\end{frame}

\begin{frame}[t,fragile]{Asignación de movimiento}
\begin{itemize}
  \item Un \cppid{shared\_ptr} soporta \textmark{asignación de movimiento}.
    \begin{itemize}
      \item Se transfiere la propiedad del bloque de memoria.
    \end{itemize}
\begin{lstlisting}
auto p = std::make_shared<double>{42.0};
auto q = std::make_shared<double>{99.0};
p = std::move(q);
\end{lstlisting}

  \mode<presentation>{\vfill\pause}
  \item Un \cppid{shared\_ptr} también se puede \textmark{asignar por movimiento}
        a partir de un \cppid{unique\_ptr}.
\begin{lstlisting}[escapechar=@]
auto p = std::make_shared<double>{42.0};
auto q = std::make_unique<double>{99.0};
p = std::move(q);
@\pause@
std::unique_ptr<cuenta> crea_cuenta();
std::unique_ptr<cuenta> u;
u = crea_cuenta();
std::shared_ptr<cuenta> s;
s = crea-cuenta();
\end{lstlisting}

\end{itemize}
\end{frame}

\subsection{Acceso}

\begin{frame}[t,fragile]{Acceso al puntero interno}
\begin{itemize}
  \item La operación \cppid{get()} permite acceder al \textmark{puntero primitivo}
        al bloque de memoria dinámica.
    \begin{itemize}
      \item El bloque \textmark{sigue siendo gestionado} por el \cppid{shared\_ptr} y
            se \textbad{libera} al destruir el puntero.
    \end{itemize}
\begin{lstlisting}
void imprime(punto * p);
//...
auto p = std::make_shared<punto>(3.5, 4.2);
imprime(p.get());
\end{lstlisting}

  \mode<presentation>{\vfill\pause}
  \item Un \cppid{shared\_ptr} es \textemph{explíctamente} convertible a \cppkey{bool}.
\begin{lstlisting}
auto p = std::make_shared<punto>(3.5, 4.2);
bool valido = p; // Error. No se puede convertir implícitamente
bool valido = static_cast<bool>(p); // OK
\end{lstlisting}

  \mode<presentation>{\vfill\pause}
  \item Pero se puede convertir \textemph{implícitamente} a \cppkey{bool}
        en conversiones contextuales.
\begin{lstlisting}
if (p) { /*...*/ } // OK. Conversión contextual
if (!p) { /*...*/ } // OK. Conversión contextual
\end{lstlisting}

\end{itemize}
\end{frame}

\begin{frame}[t,fragile]{Acceso al contador de referencias}
\begin{itemize}
  \item La operación \cppid{use\_count()} da acceso al contador de referencias.

\mode<presentation>{\vfill\pause}
\begin{lstlisting}[escapechar=@]
  auto p = std::make_shared<double>(42.0);
  std::cout << "count(p)= " << p.use_count() << "\n"; // 1@\pause@
  auto q = p;
  std::cout << "count(p)= " << p.use_count() << "\n"; // 2
  std::cout << "count(q)= " << q.use_count() << "\n"; // 2@\pause@
  auto r = std::move(p);
  std::cout << "count(p)= " << p.use_count() << "\n"; // 0
  std::cout << "count(q)= " << q.use_count() << "\n"; // 2
  std::cout << "count(r)= " << r.use_count() << "\n"; // 2@\pause@
  q = nullptr;
  std::cout << "count(p)= " << p.use_count() << "\n"; // 0
  std::cout << "count(q)= " << q.use_count() << "\n"; // 0
  std::cout << "count(r)= " << r.use_count() << "\n"; // 1@\pause@
  r = nullptr;
  std::cout << "count(p)= " << p.use_count() << "\n"; // 0
  std::cout << "count(q)= " << q.use_count() << "\n"; // 0
  std::cout << "count(r)= " << r.use_count() << "\n"; // 0
\end{lstlisting}
\end{itemize}
\end{frame}

\begin{frame}[t,fragile]{Acceso al contenido}
\begin{itemize}
  \item Están definidos los operadores \cppkey{*} y \cppkey{->}:
\begin{lstlisting}
auto p = std::make_shared<punto>(3.5, 4.2);
punto pos = *p;
double x = p->x();
\end{lstlisting}

  \mode<presentation>{\vfill\pause}
  \item Si el \cppid{shared\_ptr} es de tipo array se puede usar el operador \cppkey{[]}.
\begin{lstlisting}
auto v = std::make_shared<punto[]>(50);
v[10].x = 3.5;
\end{lstlisting}

\end{itemize}
\end{frame}
