\section{Copia por defecto}

\begin{frame}[fragile]{Operaciones de copia generadas}
\begin{itemize}
  \item Por cada tipo se generan de forma automática dos operaciones de copia:
    \begin{itemize}
      \item Construcción de copia.
\begin{lstlisting}
T y;
T x{y}; // Construcción de copia
\end{lstlisting}
      \item Asignación de copia.
\begin{lstlisting}
T x, y;
x = y; // Asignación de copia
\end{lstlisting}
    \end{itemize}
  \item Implementación por defecto:
    \begin{itemize}
      \item Invocación (recursiva) de operaciones de copia para cada miembro.
      \item La copia de tipos primitivos es la copia del valor.
    \end{itemize}
  \item ¿Es esto lo adecuado para nuestro \cppid{vector}?
\end{itemize}
\end{frame}

\begin{frame}[fragile]
\begin{block}{main.cpp}
\lstinputlisting{03-copymove/vec3/main.cpp}
\end{block}
\begin{lstlisting}[style=terminal]
0 0 5 0 3.5 
0 0 5 0 3.5 
*** glibc detected *** ./test: double free or corruption (fasttop): 
0x0000000001022010 ***
\end{lstlisting}
\end{frame}

\begin{frame}[fragile]
\begin{lstlisting}[style=terminal]
valgrind ./test
\end{lstlisting}
\begin{lstlisting}[style=terminal,basicstyle=\tiny\ttfamily]
==22680== Memcheck, a memory error detector
==22680== Copyright (C) 2002-2011, and GNU GPL'd, by Julian Seward et al.
==22680== Using Valgrind-3.7.0 and LibVEX; rerun with -h for copyright info
==22680== Command: ./test
==22680== 
0 0 5 0 3.5 
0 0 5 0 3.5 
==22680== Invalid free() / delete / delete[] / realloc()
==22680==    at 0x4C2A09C: operator delete[](void*) (in /usr/lib/valgrind/vgpreload_memcheck-amd64-linux.so)
==22680==    by 0x400C66: vector::~vector() (in /home/jdaniel/Dropbox/cursos/intro-cpp11/init/vec2/test)
==22680==    by 0x400BB5: main (in /home/jdaniel/Dropbox/cursos/intro-cpp11/init/vec2/test)
==22680==  Address 0x5a06040 is 0 bytes inside a block of size 40 free'd
==22680==    at 0x4C2A09C: operator delete[](void*) (in /usr/lib/valgrind/vgpreload_memcheck-amd64-linux.so)
==22680==    by 0x400C66: vector::~vector() (in /home/jdaniel/Dropbox/cursos/intro-cpp11/init/vec2/test)
==22680==    by 0x400BA9: main (in /home/jdaniel/Dropbox/cursos/intro-cpp11/init/vec2/test)
==22680== 
==22680== 
==22680== HEAP SUMMARY:
==22680==     in use at exit: 0 bytes in 0 blocks
==22680==   total heap usage: 1 allocs, 2 frees, 40 bytes allocated
==22680== 
==22680== All heap blocks were freed -- no leaks are possible
==22680== 
==22680== For counts of detected and suppressed errors, rerun with: -v
==22680== ERROR SUMMARY: 1 errors from 1 contexts (suppressed: 2 from 2)
\end{lstlisting}
\end{frame}

\begin{frame}
\begin{itemize}
  \item \alert{Problema}: Al realizar la copia, se ha copiado la dirección del array.
    \begin{itemize}
      \item Los dos vectores están compartiendo el mismo array.
      \item Se está desasignando dos veces un mismo bloque de memoria.
    \end{itemize}
\end{itemize}
\begin{tikzpicture}
\tikzset{
    bloque/.style={rectangle,draw=black, top color=white, bottom color=blue!50,
                   very thick, inner sep=0.5em, minimum size=0.6cm, text centered, font=\tiny},
    flecha/.style={->, >=latex', shorten >=1pt, thick},
    etiqueta/.style={text centered, font=\tiny} 
}  
\node[bloque] (bsize) {5};
\node[bloque,right=0cm of bsize] (bptr) { };
\node[bloque,right=0.5cm of bptr] (v0) {0.0};
\node[bloque,right=0cm of v0] (v1) {0.0};
\node[bloque,right=0cm of v1] (v2) {5.0};
\node[bloque,right=0cm of v2] (v3) {0.0};
\node[bloque,right=0cm of v3] (v4) {3.5};
\draw[flecha] (bptr) -- (v0);
\node[etiqueta, left=0.1cm of bsize] {v1:};
\node[etiqueta, above=0cm of bsize] {tam};

\node[bloque, below=1cm of bsize] (bsize2) {5};
\node[bloque,right=0cm of bsize2] (bptr2) { };
\draw[flecha] (bptr2) -- (v0);
\node[etiqueta, left=0.1cm of bsize2] {v2:};

\end{tikzpicture}
\end{frame}

\subsection{Operaciones de copia}

\begin{frame}[fragile]{Constructor de copia}
\begin{itemize}
  \item El \alert{constructor de copia} se invoca cuando se construye un objeto
        a partir de otro.
    \begin{itemize}
      \item Toma un argumento referencia constante al tipo.
\begin{lstlisting}
vector(const vector &);
\end{lstlisting}
    \end{itemize}
  \item Se puede suprimir la generación automática del constructor de copia.
\begin{lstlisting}
vector(const vector &) = delete;
\end{lstlisting}
  \item El constructor de copia se invocará en definiciones del tipo:
\begin{lstlisting}
vector w {v};
vector w = v;
vector w(v);
\end{lstlisting}
\end{itemize}
\end{frame}

\begin{frame}{Implementación de constructor de copia}
\begin{block}{vector.cpp}
\lstinputlisting[firstline=17,lastline=22]{03-copymove/vec4/vector.cpp}
\end{block}
\end{frame}

\begin{frame}[fragile]{Operador de asignación de copia}
\begin{itemize}
  \item El \alert{operador de asignación de copia} se invoca cuando se asigna un objeto a otro.
    \begin{itemize}
      \item Toma un argumento referencia constante al tipo.
      \item Devuelve una referencia al objeto asignado.
      \item Debe ser una función miembro de la clase.
\begin{lstlisting}
vector & operator=(const vector &);
\end{lstlisting}
    \end{itemize}
  \item Se puede suprimir la generación automática del operador de asignación de copia.
\begin{lstlisting}
vector & operator=(const vector &) = delete;
\end{lstlisting}
  \item El constructor de copia se invocará en definiciones del tipo:
\begin{lstlisting}
w = v;
\end{lstlisting}
\end{itemize}
\end{frame}

\begin{frame}{Implementación de operador de asignación de copia}
\begin{block}{vector.cpp}
\lstinputlisting[firstline=24,lastline=31]{03-copymove/vec4/vector.cpp}
\end{block}
\end{frame}

\begin{frame}{El puntero \emph{this}}
\begin{itemize}
  \item \cppkey{this} es una palabra reservada que se puede evaluar dentro de cualquier función miembro de una clase.
  \item Se evalúa a la dirección de memoria del objeto para el que se está ejecutando la función miembro.
  \item La expresión \cppkey{return *this} en el operador de asignación de copia devuelve una referencia la objeto.
  \item ¿Por qué?
    \begin{itemize}
      \item v1 = v2 = v3;
    \end{itemize}
\end{itemize}
\end{frame}


