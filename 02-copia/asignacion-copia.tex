\section{Operador de asignación de copia}

\begin{frame}[t,fragile]{Operador de asignación de copia}
\begin{itemize}
  \item El \textgood{operador de asignación de copia} se invoca cuando
        \textemph{se asigna un objeto} a otro \textmark{del mismo tipo}.
    \begin{itemize}
      \item Toma un argumento referencia constante al tipo.
      \item Devuelve una referencia al objeto asignado.
      \item Debe ser una función miembro de la clase.
    \end{itemize}
\begin{lstlisting}
vecnum & operator=(const vecnum &);
\end{lstlisting}

  \mode<presentation>{\vfill\pause}
  \item Se puede suprimir la generación automática del operador de asignación de copia.
\begin{lstlisting}
vecnum & operator=(const vecnum &) = delete;
\end{lstlisting}

  \mode<presentation>{\vfill\pause}
  \item El constructor de copia se invocará en expresiones del tipo:
\begin{lstlisting}
w = v; // w.operator=(v);
\end{lstlisting}
\end{itemize}
\end{frame}

\begin{frame}[t,fragile]{Implementación de operador de asignación de copia}
\begin{block}{vector.cpp}
\begin{lstlisting}
vecnum & vecnum::operator=(const vecnum & v) {
  tamanyo_ = v.tamanyo_;
  delete []buffer_;
  buffer_ = new double[v.tamanyo_];
  std::copy_n(v.buffer_, v.tamanyo_, buffer_);
  return *this;
}
\end{lstlisting}
\end{block}

\mode<presentation>{\vfill}
\begin{itemize}
  \item Copiar el tamaño.
  \item Liberar el búffer anterior.
  \item Reservar un bloque de memoria para el nuevo búfer.
  \item Copiar elemento a elemento del buffer anterior al nuevo.
  \item Devolver una referencia al objeto acutal (\cppkey{*this}).
\end{itemize}
\end{frame}

\begin{frame}[t,fragile]{El puntero \textbf{this}}
\begin{itemize}
  \item \cppkey{this} es una palabra reservada que se puede evaluar
        \textmark{dentro} de cualquier \textgood{función miembro} de una clase.

  \mode<presentation>{\vfill\pause}
  \item Se evalúa a la \textmark{dirección de memoria} del objeto para el que se 
        está ejecutando la función miembro.
    \begin{itemize}
      \item \cppkey{this} es un puntero.
    \end{itemize}

  \mode<presentation>{\vfill\pause}
  \item La expresión \cppkey{return *this} en el operador de asignación de copia devuelve una referencia la objeto.

  \mode<presentation>{\vfill\pause}
  \item \textmark{¿Por qué?}
\begin{lstlisting}[escapechar=@]
v1 = v2 = v3; @\pause@
v1.operator=(v2=v3); @\pause@
v1.operator=(v.operator=(v3));
\end{lstlisting}
    \begin{itemize}
      \item Asociativo por la derecha.
    \end{itemize}
\end{itemize}
\end{frame}

\begin{frame}[t,fragile]{Optimizando la auto-asignación}
\begin{itemize}
  \item Caso particular: \textmark{auto-asignación}
\begin{lstlisting}
v = v; // v.operator=(v);
\end{lstlisting}
    \begin{itemize}
      \item Debería no hacerse nada.
    \end{itemize}

  \mode<presentation>{\vfill\pause}
\begin{block}{Constructor de copia}
\begin{lstlisting}
vecnum & vecnum::operator=(const vecnum & v) {
  if (this == &v) return *this;
  tamanyo_ = v.tamanyo_;
  delete []buffer_;
  buffer_ = new double[v.tamanyo_];
  std::copy_n(v.buffer_, v.tamanyo_, buffer_);
  return *this;
}
\end{lstlisting}
\end{block}
\end{itemize}
\end{frame}
