\section{Constructor de copia}

\begin{frame}[fragile]{Constructor de copia}
\begin{itemize}
  \item El \textgood{constructor de copia} se invoca cuando se 
        \textemph{construye un objeto} a partir de \textmark{otro del mismo tipo}.
    \begin{itemize}
      \item Toma un argumento referencia constante al tipo.
\begin{lstlisting}
vecnum(const vector &);
\end{lstlisting}
    \end{itemize}

  \mode<presentation>{\vfill\pause}
  \item Se puede suprimir la generación automática del constructor de copia.
\begin{lstlisting}
vecnum(const vecnum &) = delete;
\end{lstlisting}

  \mode<presentation>{\vfill\pause}
  \item El constructor de copia se invocará en definiciones del tipo:
\begin{lstlisting}
vecnum w {v}; // Iniciación directa
vecnum w(v);  // Iniciación directa. Sintaxis tradicional.
vecnum w = v; // Iniciación de copia. No es asignación.
\end{lstlisting}
\end{itemize}
\end{frame}

\begin{frame}[t]{Implementación de constructor de copia}
\begin{block}{vecnum.cpp}
\lstinputlisting[firstline=11,lastline=15]{ejemplos/02-copia/vecnum-copy/vecnum.cpp}
\end{block}

\mode<presentation>{\vfill}
\begin{enumerate}
  \item Copia el tamaño.
  \item Reserva memoria para un nuevo búfer.
  \item Copia los elementos del búfer origen en el búfer destino.
    \begin{itemize}
      \item \cppid{copy\_n(p,n,q)}: Copia \cppid{n} elementos consecutivos almancenados
            a partir de \cppid{p} en posiciones consecutivas a partir de \cppid{q}.
    \end{itemize} 
\end{enumerate}
\end{frame}
