\section{Herencia}

\subsection{Derivación de clases}

\begin{frame}[t,fragile]{Herencia}
\begin{itemize}
  \item La definición de una clase puede especificar que 
        \textmark{hereda} de otra clase.
\end{itemize}

\begin{columns}[T]

\column{.1\textwidth}
\column{.5\textwidth}
\begin{lstlisting}[escapechar=@]
class base {
public:
  // miembros públicos
private:
  // miembros privados
};

class derivado : @\textbad{public base}@ {
public:
  // miembros públicos
private:
  // miembros privados
};

base b; // objeto de tipo base
derivado d; // Objeto de tipo derivado
\end{lstlisting}

\column{.5\textwidth}
\begin{itemize}
  \item El objeto \cppid{d} tiene todos los miembros del tipo \cppid{base}
        y todos los miembros del tipo \cppid{derivado}.
\end{itemize}
\end{columns}

\end{frame}

\subsection{Aplicación práctica de la herencia}

\begin{frame}[t]{Ejemplo: Figuras geométricas}
\begin{itemize}
  \item \textmark{Caso}: Biblioteca de figuras geométricas

  \mode<presentation>{\vfill\pause}
  \item \textmark{Abstracciones}:
    \begin{itemize}
      \mode<presentation>{\vfill\pause}
      \item \cppid{punto}: 
        \begin{itemize}
          \item \textmark{Datos}: Coordenadas \cppid{x} e \cppid{y}.
          \item \textmark{Operaciones}: Inserción en flujo.
        \end{itemize}
      \mode<presentation>{\vfill\pause}
      \item \cppid{figura}: 
        \begin{itemize}
          \item \textmark{Datos}: Un \cppid{punto} que representa la \cppid{posición}.
          \item \textmark{Operaciones}: \cppid{desplazar}.
        \end{itemize}

      \mode<presentation>{\vfill\pause}
      \item \cppid{circulo}: 
        \begin{itemize}
          \item \textmark{Datos}: Una \cppid{posición} y un \cppid{radio}.
          \item \textmark{Operaciones}: \cppid{desplazar}, calcular el \cppid{area}
                e insertar en un flujo.
        \end{itemize}
    \end{itemize}
\end{itemize}
\end{frame}

\mode<presentation>{

\begin{frame}[t]
\begin{block}{punto.hpp}
\lstinputlisting[lastline=20]{ejemplos/06-dynpol/geom/punto.hpp}
\end{block}
\end{frame}

\begin{frame}[t]
\begin{block}{punto.hpp}
\lstinputlisting[firstline=22]{ejemplos/06-dynpol/geom/punto.hpp}
\end{block}
\end{frame}

}

\mode<article>{
\begin{block}{punto.hpp}
\lstinputlisting{ejemplos/06-dynpol/geom/punto.hpp}
\end{block}
}

\begin{frame}[t]
\begin{block}{figura.hpp}
\lstinputlisting[lastline=17]{ejemplos/06-dynpol/geom/figura.hpp}
\end{block}
\end{frame}

\begin{frame}[t]
\begin{block}{figura.hpp}
\lstinputlisting[firstline=19]{ejemplos/06-dynpol/geom/figura.hpp}
\end{block}
\end{frame}

\begin{frame}[t]
\begin{block}{figura.cpp}
\lstinputlisting{ejemplos/06-dynpol/geom/figura.cpp}
\end{block}
\end{frame}

\mode<presentation>{

\begin{frame}[t]
\begin{block}{circulo.hpp}
\lstinputlisting[lastline=19]{ejemplos/06-dynpol/geom/circulo.hpp}
\end{block}
\end{frame}

\begin{frame}[t]
\begin{block}{circulo.hpp}
\lstinputlisting[firstline=20]{ejemplos/06-dynpol/geom/circulo.hpp}
\end{block}
\end{frame}

}

\mode<article>{
\begin{frame}[t]

\begin{block}{circulo.hpp}
\lstinputlisting{ejemplos/06-dynpol/geom/circulo.hpp}
\end{block}
\end{frame}

}

\begin{frame}[t]
\begin{block}{circulo.cpp}
\lstinputlisting{ejemplos/06-dynpol/geom/circulo.cpp}
\end{block}
\end{frame}

\subsection{Conversiones y herencia}

\begin{frame}[t,fragile]{Conversiones de tipo}
\begin{itemize}
  \item Se puede \textmark{copiar} 
        un \textgood{objeto derivado} en un \textgood{objeto base}.
    \begin{itemize}
      \item Solamente se copia la porción de la clase base.
    \end{itemize}
\begin{lstlisting}
circulo c{punto{1.5, 1.5}, 3.0};
figura f = c;
\end{lstlisting}

  \mode<presentation>{\vfill\pause}
  \item Se puede \textmark{pasar por valor} 
        \textgood{un objeto derivado} donde se espera un \textgood{objeto base}.
\begin{lstlisting}
void imprime(figura f);
circulo c{punto{1.5, 1.5}, 3.0};
imprime(c);
\end{lstlisting}

\end{itemize}
\end{frame}

\begin{frame}[t,fragile]{Referencias y herencia}
\begin{itemize}

  \item Se puede \textmark{convertir} 
        una \textemph{referencia} a una \textgood{clase derivada}
        a una \textemph{referencias} a una \textgood{clase base}.
\begin{lstlisting}
circulo c{punto{1.5, 1.5}, 3.0};
figura & f = c;
const figura & g = c;
\end{lstlisting}

  \mode<presentation>{\vfill\pause}
  \item Se puede \textmark{pasar por referencia} 
        \textgood{un objeto derivado} donde se espera un \textgood{objeto base}.
\begin{lstlisting}
void mueve_origen(figura & f);
void imprime(const figura & f);
circulo c{punto{1.5, 1.5}, 3.0};
mueve_origen(c);
imprime(c);
\end{lstlisting}

\end{itemize}
\end{frame}

\begin{frame}[t,fragile]{Punteros y herencia}
\begin{itemize}

  \item Aplicable también a \textgood{punteros primitivos}.
\begin{lstlisting}
circulo c{punto{1.5, 1.5}, 3.0};
figura * pc = &c;
const figura * qc = &c;
\end{lstlisting}

  \mode<presentation>{\vfill\pause}
  \item Aplicable también a \textgood{punteros elegantes}.
\begin{lstlisting}
auto pc = std::make_unique<circulo>(punto{0.5, 0.5}, 3.0);
std::unique_ptr<figura> pf = std::move(pc);
auto pd = std::make_unique<circulo>(punto{1.5, 1.5}, 0.5);
std::unqiue_ptr<const figura> qf = std::move(pd);

auto sc = std::make_shared<circulo>(punto{0.5, 0.5}, 3.0);
std::shared_ptr<figura> sf = sc;
std::shared_ptr<const figura> tf = sc;
\end{lstlisting}

\end{itemize}
\end{frame}

\subsection{Funciones miembro}

\begin{frame}[t,fragile]{Añadiendo más clases derivadas}
\begin{itemize}
  \item Se puede añadir ahora otra clase para representar un \textmark{rectángulo}.
\begin{lstlisting}
class rectangulo : public figura {
public:
  rectangulo(punto a, punto b) 
      : figura{a}, 
        ancho_{b.x-a.x},
        alto_{b.y-a.y} {}
  //...
private:
  double ancho_;
  double alto_;
};
\end{lstlisting}

  \mode<presentation>{\vfill\pause}
  \item \textmark{Nuevo}: Se podría añadir una operación \cppid{centro()}
        que devuelva las coordenadas del centro de la figura.
\end{itemize}
\end{frame}

\begin{frame}[t,fragile]{Acceso a miembros}
\begin{itemize}
  \item La implementación de una función en una \textgood{clase derivada}
        \textmark{puede acceder} a cualquier \textemph{miembro público} de la clase base.
\begin{lstlisting}
class circulo : public figura {
public:
  //...
  [[nodiscard]] punto centro() const { return posicion(); }
  //...
};
\end{lstlisting}

  \mode<presentation>{\vfill\pause}
  \item Pero \textbad{no puede acceder} a los \textemph{miembros privados}
        de la clase base.
\begin{lstlisting}
class circulo : public figura {
public:
  //...
  [[nodiscard]] punto centro() const { return posicion_; } // Error. Privado
  //...
};
\end{lstlisting}

\end{itemize}
\end{frame}

\begin{frame}[t,fragile]{Implementación por defecto}
\begin{itemize}
  \item \textmark{Alternativa}: Definir \cppid{centro()} 
        en la clase base (\cppid{figura})
        y se \textmark{hereda} en todas las clases derivadas.
\begin{lstlisting} 
class figura {
public:
  [[nodiscard]] punto centro() const { return posicion_; } // OK
  //...
};
\end{lstlisting}

  \mode<presentation>{\vfill\pause}
  \item \textbad{Problema}: Hay que redefinir \cppid{centro()}
        en \cppid{rectangulo} usando la operación \cppid{centro()} de \cppid{figura}.
\begin{lstlisting}[escapechar=@]
class rectangulo : public figura {
public:
  [[nodiscard]] punto centro() const { // Problema: Recursividad infinita.
    return @\color{red}centro()@ + desplazamiento{ancho_/2,alto_/2};
  }
  //...
}
\end{lstlisting}

\end{itemize}
\end{frame}

\begin{frame}[t,fragile]{Acceso a miembro de la base}
\begin{itemize}
  \item En una \textgood{clase derivada}
        se puede cualificar el acceso a un miembro de \textgood{la clase base}.
    \begin{itemize}
      \item Permite resolver la ambigüedad y seleccionar la versión de la base.
    \end{itemize}
\begin{lstlisting}[escapechar=@]
class rectangulo : public figura {
public:
  //...
  [[nodiscard]] punto centro() const {
    return @\color{red}figura::centro()@ + desplazamiento{ancho_/2,alto_/2};
  }
\end{lstlisting}
\end{itemize}
\end{frame}

\subsection{Herencia y sobrecarga}

\begin{frame}[t,fragile]{Sobrecarga en clase derivada}
\begin{itemize}
  \item La \textgood{sobrecarga de funciones} \textbad{no actúa} sobre funciones miembro 
        de \textmark{distintas clases}.

\mode<presentation>{\vfill\pause}
\begin{lstlisting}
class figura {
public:
  void desplaza(desplazamiento d) noexcept { posicion_ += d; }
  //...
};

class circulo : public figura {
public:
  void desplaza(double delta) noexcept { figura::desplaza({delta,delta}); }
  //...
};

//...
circulo c{punto{1.0,1.0}, 2.0};
c.desplaza(3.0); //OK
c.desplaza(desplazamiento{2.0,3.0}); // Error: no puede covertir a double
\end{lstlisting}
\end{itemize}
\end{frame}

\begin{frame}[t,fragile]{Problemas \emph{silenciosos}}
\begin{itemize}
  \item El conflicto entre \textmark{sobrecarga} y \textmark{herencia}
        puede dar lugar a \textbad{errores} difíciles de detectar.
\begin{lstlisting}
class A {
public:
  void f(int x) { std::cout << "int\n"; }
};

class B : public A {
public:
  void f(double x) { std::cout << "double\n"; }
};

B b;
b.f(1.0); // Imprime double
b.f(1); // Sorpresa: Imprime double
\end{lstlisting}
\end{itemize}
\end{frame}

\begin{frame}[t,fragile]{Herencia de funciones sobrecargadas}
\begin{itemize}
  \item Se puede utilizar una \textgood{declaración de uso} para \textmark{añadir}
        todas las funciones sobrecargadas de una clase base.

\end{itemize}

\mode<presentation>{\vfill\pause}
\begin{lstlisting}
class figura {
public:
  void desplaza(desplazamiento d) noexcept { posicion_ += d; }
  //...
};

class circulo : public figura {
public:
  using figura::desplaza;
  void desplaza(double delta) noexcept { figura::desplaza({delta,delta}); }
  //...
};

//...
circulo c{punto{1.0,1.0}, 2.0};
c.desplaza(3.0); //OK
c.desplaza(desplazamiento{2.0,3.0}); // OK
\end{lstlisting}
\end{frame}

\begin{frame}[t,fragile]{Herencia de constructores}
\begin{itemize}
  \item Los constructores de la \textgood{clase base} no están automáticamente
        disponibles en la \textgood{clase derivada}.

\mode<presentation>{\vfill\pause}
\begin{lstlisting}
class diagrama : public std::vector<std::unique_ptr<figura>> {
public:
  void desplaza(desplazamiento d) {
    for (auto & x : *this) { if (x) { x->desplaza(d); } }
  }
};
//...
diagrama d(3); // No hay constructor
\end{lstlisting}

  \mode<presentation>{\vfill\pause}
  \item Normalmente hay \textgood{buenas razones}:
    \begin{itemize}
      \item Si hay datos miembros en la clase derivada hay que iniciarlos.
      \item Postcondiciones adicionales.
    \end{itemize}

\end{itemize}
\end{frame}

\begin{frame}[t,fragile]{Declaración de uso y constructores}
\begin{itemize}
  \item Se pueden heredar todos los constructores con una \textgood{declaración de uso}.
\begin{lstlisting}
class diagrama : public std::vector<std::unique_ptr<figura>> {
public:
  using std::vector<std::unique_ptr<figura>>::vector; 
};
//...
diagrama d(3); // OK
\end{lstlisting}
\end{itemize}
\end{frame}

\subsection{Control de acceso}

\begin{frame}[t,fragile]{Niveles de visibilidad}
\begin{itemize}
  \item Aparece un tercer nivel de visibilidad en las clases además de
        \cppkey{public} y \cppkey{private} $\Rightarrow$ \cppkey{protected}.
    \begin{itemize}
      \item \cppkey{private}:
            Accesible por funciones miembro de la propia clase.
      \item \cppkey{protected}:
            Accesible por funciones miembro de la propia clase y de clases derivadas.
      \item \cppkey{public}:
            Accesible por funciones miembro de la propia clase, clases derivadas y
            por usuarios de la clase.
    \end{itemize}

  \mode<presentation>{\vfill\pause}
  \item El acceso \textemph{protegido} permite definir una intefaz
        solamente utilizable por las clases derivadas.
\end{itemize}
\end{frame}

\begin{frame}[t,fragile]{Miembros protegidos}
\begin{itemize}
  \item Un miembro se hace \textemph{protegido} si se necesita usar
        desde una clase derivada pero no se desea exponer
        públicamente.

\mode<presentation>{\vfill\pause}
\begin{lstlisting}[escapechar=@]
class figura {
//...
@\color{red}protected@:
  double posicion_;
};

class rectangulo {
public:
  void refleja_x() noexcept { @\color{red}posicion\_.x@ = -@\color{red}posicion\_.x@ - ancho_; }
private:
  double ancho_, double alto_;
};
\end{lstlisting}
\end{itemize}
\end{frame}

\begin{frame}[t,fragile]{Herencia pública}
\begin{itemize}
  \item Relaciones del tipo \textmark{es-un}.
\end{itemize}

\mode<presentation>{\vspace{-1em}}
\begin{columns}[T]
\column{.6\textwidth}
\begin{lstlisting}
class figura {
public:
  void desplaza(desplazamiento d) noexcept;
protected:
  double posicion_;
};
\end{lstlisting}
\column{.5\textwidth}
\begin{lstlisting}
class circulo : public figura { 
public:
  punto centro() const noexcept;
};
\end{lstlisting}
\end{columns}

  \mode<presentation>{\vspace{-1em}}
  \mode<presentation>{\vfill\pause}
\begin{itemize}
  \item Los miembros públicos de la base 
        son accesibles por cualquier función
\begin{lstlisting}
circulo c{punto{1.0,1.0}, 3.0};
c.desplaza(); // OK. Accesible siempre.
\end{lstlisting}

  \mode<presentation>{\vfill\pause}
  \item Los miembros protegidos de la base 
        son accesibles por los miembros de la clase derivada.
\begin{lstlisting}
punto circulo::centro() const noexcept { return posicion_; } // OK
circulo c{punto{1.0,1.0}, 3.0};
auto c = c.posicion_; // Error. No accesible.
\end{lstlisting}

\end{itemize}
\end{frame}

\begin{frame}[t,fragile]{Herencia privada}
\begin{itemize}
  \item La clase base es un detalle de implementación de la derivada.
    \begin{itemize}
      \item Útil para restringir la interfaz de una clase.
    \end{itemize}
\begin{lstlisting}
class diagrama : private std::vector<std::shared_ptr<figura>> {
public:
  using std::vector<std::shared_ptr<figura>>::vector;
  void agrega_circulo(punto c, double r); 
  void agrega_rectangulo(punto p1, punto p2);
  void desplaza(desplazamiento d);
};
\end{lstlisting}

  \mode<presentation>{\vfill\pause}
  \item Los miembros públicos y protegidos de la base
        son accesibles por los miembros de la clase derivada.
\begin{lstlisting}
void diagrama::agrega_circulo(punto c, double r) {
  push_back(std::make_unique<circulo>(c, r));
}
\end{lstlisting}
\end{itemize}
\end{frame}

\begin{frame}[t,fragile]{Herencia protegida}
\begin{itemize}
  \item La clase base es un detalle de implementación 
        de la clase derivada.
    \begin{itemize}
      \item La clase derivada es una clase intermedia en la jerarquía.
    \end{itemize}
\begin{lstlisting}
class diagrama : protected std::vector<std::shared_ptr<figura>> {/*...*/  };

class diagrama_anotado: public diagrama { 
public:
  void agrega_anotacion(std::string texto);
};
\end{lstlisting}

  \mode<presentation>{\vfill\pause}
  \item Los miembros públicos y protegidos de la base
        son accesibles por los miembros de la clase derivada
        y sus descendientes.
\begin{lstlisting}
void agrega_anotacion(const std::string & texto) {
  push_back(std::make_unique<anotacion>(texto));
}
\end{lstlisting}

\end{itemize}
\end{frame}
