\section{Herencia}

\subsection{Derivación de clases}

\begin{frame}[t,fragile]{Herencia}
\begin{itemize}
  \item La definición de una clase puede especificar que 
        \textmark{hereda} de otra clase.
\end{itemize}

\begin{columns}[T]

\column{.1\textwidth}
\column{.5\textwidth}
\begin{lstlisting}[escapechar=@]
class base {
public:
  // miembros públicos
private:
  // miembros privados
};

class derivado : @\textbad{public base}@ {
public:
  // miembros públicos
private:
  // miembros privados
};

base b; // objeto de tipo base
derivado d; // Objeto de tipo derivado
\end{lstlisting}

\column{.5\textwidth}
\begin{itemize}
  \item El objeto \cppid{d} tiene todos los miembros del tipo \cppid{base}
        y todos los miembros del tipo \cppid{derivado}.
\end{itemize}
\end{columns}

\end{frame}

\subsection{Aplicación práctica de la herencia}

\begin{frame}[t]{Ejemplo: Figuras geométricas}
\begin{itemize}
  \item \textmark{Caso}: Biblioteca de figuras geométricas

  \mode<presentation>{\vfill\pause}
  \item \textmark{Abstracciones}:
    \begin{itemize}
      \mode<presentation>{\vfill\pause}
      \item \cppid{punto}: 
        \begin{itemize}
          \item \textmark{Datos}: Coordenadas \cppid{x} e \cppid{y}.
          \item \textmark{Operaciones}: Inserción en flujo.
        \end{itemize}
      \mode<presentation>{\vfill\pause}
      \item \cppid{figura}: 
        \begin{itemize}
          \item \textmark{Datos}: Un \cppid{punto} que representa la \cppid{posición}.
          \item \textmark{Operaciones}: \cppid{desplazar}.
        \end{itemize}

      \mode<presentation>{\vfill\pause}
      \item \cppid{circulo}: 
        \begin{itemize}
          \item \textmark{Datos}: Una \cppid{posición} y un \cppid{radio}.
          \item \textmark{Operaciones}: \cppid{desplazar}, calcular el \cppid{area}
                e insertar en un flujo.
        \end{itemize}
    \end{itemize}
\end{itemize}
\end{frame}

\begin{frame}[t]
\begin{block}{figura.hpp}
\lstinputlisting[lastline=17]{ejemplos/06-dynpol/geom/figura.hpp}
\end{block}
\end{frame}

\begin{frame}[t]
\begin{block}{figura.hpp}
\lstinputlisting[firstline=19]{ejemplos/06-dynpol/geom/figura.hpp}
\end{block}
\end{frame}

\begin{frame}[t]
\begin{block}{figura.cpp}
\lstinputlisting{ejemplos/06-dynpol/geom/figura.cpp}
\end{block}
\end{frame}

\begin{frame}[t]
\begin{block}{circulo.hpp}
\lstinputlisting{ejemplos/06-dynpol/geom/circulo.hpp}
\end{block}
\end{frame}

\begin{frame}[t]
\begin{block}{circulo.cpp}
\lstinputlisting{ejemplos/06-dynpol/geom/circulo.cpp}
\end{block}
\end{frame}

\subsection{Conversiones y herencia}

\begin{frame}[t,fragile]{Conversiones de tipo}
\begin{itemize}
  \item Se puede \textmark{copiar} 
        un \textgood{objeto derivado} en un \textgood{objeto base}.
    \begin{itemize}
      \item Solamente se copia la porción de la clase base.
    \end{itemize}
\begin{lstlisting}
circulo c{punto{1.5, 1.5}, 3.0};
figura f = c;
\end{lstlisting}

  \mode<presentation>{\vfill\pause}
  \item Se puede \textmark{pasar por valor} 
        \textgood{un objeto derivado} donde se espera un \textgood{objeto base}.
\begin{lstlisting}
void imprime(figura f);
circulo c{punto{1.5, 1.5}, 3.0};
imprime(c);
\end{lstlisting}

\end{itemize}
\end{frame}

\begin{frame}[t,fragile]{Referencias y herencia}
\begin{itemize}

  \item Se puede \textmark{convertir} 
        una \textemph{referencia} a una \textgood{clase derivada}
        a una \textemph{referencias} a una \textgood{clase base}.
\begin{lstlisting}
circulo c{punto{1.5, 1.5}, 3.0};
figura & f = c;
const figura & g = c;
\end{lstlisting}

  \mode<presentation>{\vfill\pause}
  \item Se puede \textmark{pasar por referencia} 
        \textgood{un objeto derivado} donde se espera un \textgood{objeto base}.
\begin{lstlisting}
void mueve_origen(figura & f);
void imprime(const figura & f);
circulo c{punto{1.5, 1.5}, 3.0};
mueve_origen(c);
imprime(c);
\end{lstlisting}

\end{itemize}
\end{frame}

\begin{frame}[t,fragile]{Punteros y herencia}
\begin{itemize}

  \item Aplicable también a \textgood{punteros primitivos}.
\begin{lstlisting}
circulo c{punto{1.5, 1.5}, 3.0};
figura * pc = &c;
const figura * qc = &c;
\end{lstlisting}

  \mode<presentation>{\vfill\pause}
  \item Aplicable también a \textgood{punteros elegantes}.
\begin{lstlisting}
auto pc = std::make_unique<circulo>(punto{0.5, 0.5}, 3.0);
std::unique_ptr<figura> pf = std::move(pc);
auto pd = std::make_unique<circulo>(punto{1.5, 1.5}, 0.5);
std::unqiue_ptr<const figura> qf = std::move(pd);

auto sc = std::make_shared<circulo>(punto{0.5, 0.5}, 3.0);
std::shared_ptr<figura> sf = sc;
std::shared_ptr<const figura> tf = sc;
\end{lstlisting}

\end{itemize}
\end{frame}

\subsection{Funciones miembro}

\begin{frame}[t,fragile]{Añadiendo más clases derivadas}
\begin{itemize}
  \item Se puede añadir ahora otra clase para representar un \textmark{rectángulo}.
\begin{lstlisting}
class rectangulo : public figura {
public:
  rectangulo(punto a, punto b) 
      : figura{a}, 
        ancho_{b.x-a.x},
        alto_{b.y-a.y} {}
  //...
private:
  double ancho_;
  double alto_;
};
\end{lstlisting}

  \mode<presentation>{\vfill\pause}
  \item \textmark{Nuevo}: Se podría añadir una operación \cppid{centro()}
        que devuelva las coordenadas del centro de la figura.
\end{itemize}
\end{frame}

\begin{frame}[t,fragile]{Acceso a miembros}
\begin{itemize}
  \item La implementación de una función en una \textgood{clase derivada}
        \textmark{puede acceder} a cualquier \textemph{miembro público} de la clase base.
\begin{lstlisting}
class circulo : public figura {
public:
  //...
  [[nodiscard]] punto centro() const { return posicion(); }
  //...
};
\end{lstlisting}

  \mode<presentation>{\vfill\pause}
  \item Pero \textbad{no puede acceder} a los \textemph{miembros privados}
        de la clase base.
\begin{lstlisting}
class circulo : public figura {
public:
  //...
  [[nodiscard]] punto centro() const { return posicion_; } // Error. Privado
  //...
};
\end{lstlisting}

\end{itemize}
\end{frame}

\begin{frame}[t,fragile]{Implementación por defecto}
\begin{itemize}
  \item \textmark{Alternativa}: Definir \cppid{centro()} 
        en la clase base (\cppid{figura})
        y se \textmark{hereda} en todas las clases derivadas.
\begin{lstlisting} 
class figura {
public:
  [[nodiscard]] punto centro() const { return posicion_; } // OK
  //...
};
\end{lstlisting}

  \mode<presentation>{\vfill\pause}
  \item \textbad{Problema}: Hay que redefinir \cppid{centro()}
        en \cppid{rectangulo} usando la operación \cppid{centro()} de \cppid{figura}.
\begin{lstlisting}[escapechar=@]
class rectangulo : public figura {
public:
  [[nodiscard]] punto centro() const { // Problema: Recursividad infinita.
    return @\color{red}centro()@ + desplazamiento{ancho_/2,alto_/2};
  }
  //...
}
\end{lstlisting}

\end{itemize}
\end{frame}

\begin{frame}[t,fragile]{Acceso a miembro de la base}
\begin{itemize}
  \item En una \textgood{clase derivada}
        se puede cualificar el acceso a un miembro de \textgood{la clase base}.
    \begin{itemize}
      \item Permite resolver la ambigüedad y seleccionar la versión de la base.
    \end{itemize}
\begin{lstlisting}[escapechar=@]
class rectangulo : public figura {
public:
  //...
  [[nodiscard]] punto centro() const {
    return @\color{red}figura::centro()@ + desplazamiento{ancho_/2,alto_/2};
  }
\end{lstlisting}
\end{itemize}
\end{frame}
