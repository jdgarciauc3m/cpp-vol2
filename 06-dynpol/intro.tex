\section{Introducción}

\begin{frame}[t]{Herencia de clases}
\begin{itemize}
  \item La \textgood{herencia} permite definir nuevas clases a partir de clases
        existentes.
    \begin{itemize}
      \item Es un mecanismo de \textmark{reutilización de software}.
      \item Constituye la base de la \textmark{programación orientada a objetos}
    \end{itemize}

  \mode<presentation>{\vfill\pause}
  \item \textmark{Dos tipos}:
    \begin{itemize}

    \mode<presentation>{\vfill\pause}
    \item \textemph{Herencia de interfaz}:
          Modela una relación del tipo \textmark{es-un}.

    \mode<presentation>{\vfill\pause}
    \item \textemph{Herencia de implementación}
          Permite \textmark{reutilizar} la implementación de una clase en otra.

    \end{itemize}
\end{itemize}
\end{frame} 

\begin{frame}[t]{Herencia de interfaz}
\begin{itemize}
  \item Representa una relación de \textgood{subtipo} entre dos clases.
  
  \mode<presentation>{\vfill\pause}
  \item La clase \cppid{circulo} es un \textmark{subtipo} de la clase \cppid{figura}.
    \begin{itemize}
      \item La clase \cppid{circulo} es una clase derivada de la clase \cppid{figura}.
      \item La clase \cppid{figura} es una clase base de la clase \cppid{circulo}.
    \end{itemize}

  \mode<presentation>{\vfill\pause}
  \item La clase \textgood{derivada} \textmark{hereda} todos los miembros
        de la clase \textgood{base}.
    \begin{itemize}
      \item Todas las operaciones de la clase base se pueden aplicar a 
            la clase derivada.
    \end{itemize}

  \mode<presentation>{\vfill\pause}
  \item Un objeto de la clase \textgood{derivada} debe poderse utilizar
        en lugares donde se admite un objeto de de la clase \textgood{base}.
\end{itemize}
\end{frame}

\begin{frame}[t]{Herencia de implementación}
\begin{itemize}
  \item Permite expresar una relación de herencia que no implica \cppid{subtipo}.

  \mode<presentation>{\vfill\pause}
  \item Un objeto de la clase \textgood{empleado} 
        \textmark{tiene} un \textemph{sub-objeto}
        de la clase \textgood{persona}.

  \mode<presentation>{\vfill\pause}
  \item Un objeto de la clase \textgood{derivada} no tiene por qué
        poderse utilizar en lugares donde se admite un objeto de la clase base.

  \mode<presentation>{\vfill\pause}
  \item A veces se puede sustituir por \textemph{composición}.
\end{itemize}
\end{frame}
