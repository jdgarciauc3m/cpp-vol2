\section{Polimorfismo dinámico}

\subsection{Funciones virtuales}

\begin{frame}[t,fragile]{Invocación a funciones redefinidas}
\begin{lstlisting}
void imprime(const geom::rectangulo & r) {
  std::cout << "centro: " << r.centro() << "\n"; // rectagulo::centro()
}

void imprime(const geom::figura & f) {
  std::cout << "centro: " << f.centro() << "\n"; // figura::centro()
}

int main() {
  using namespace geom;
  rectangulo r{punto{1.5, 1.5}, punto{3.0,3.0}};
  figura & f = r;
  imprime(r);
  imprime(f);
}
\end{lstlisting}

\mode<presentation>{\vfill\pause}
\begin{lstlisting}[style=terminal]
centro: [2.25 , 2.25]
centro: [1.5 , 1.5]
\end{lstlisting}

\end{frame}

\begin{frame}[t,fragile]{Vinculación temprana frente a tardía}
\begin{itemize}
  \item Todas las funciones miembro tienen por defecto 
        \textmark{vinculación temprana} (\emph{early binding}).
    \begin{itemize}
      \item La función se selecciona en tiempo de compilación dependiendo
            del tipo estático del objeto.
    \end{itemize}

  \mode<presentation>{\vfill\pause}
  \item Se puede especificar que una función tenga
        \textmark{vinculación tardía} (\emph{late binding}).
    \begin{itemize}
      \item La función se selecciona en tiempo de ejecución dependiendo
            del tipo dinámico del objeto.
    \end{itemize}

  \mode<presentation>{\vfill\pause}
  \item La palabra reservada \cppkey{virtual} permite marcar una función
        como de \textmark{vinculación tardía}.
    \begin{itemize}
      \item Debe usarse en la la declaración, pero no en la definición.
      \item Si una clase no redefine la versión de la clase base, se hereda ésta.
      \item Si una clase la redefine se debe indicar con \cppkey{overrride}.
    \end{itemize}
\end{itemize}
\end{frame}

\begin{frame}[t,fragile]{Uso de \textbf{virtual} y \textbf{override}}
\begin{lstlisting}[escapechar=@]
class figura {
public:
  //...
  [[nodiscard]] @\color{red}virtual@ punto centro() const { return posicion_; }
  //...
};
@\pause@
class circulo : public figura {
  // Sin redefinición. Hereda figura::centro()
};
@\pause@
class rectangulo : public figura {
public:
  //...
  [[nodiscard]] punto centro() const @\color{red}override@ {
    return figura::centro() + desplazamiento{ancho_/2,alto_/2};
  }
  //...
};
\end{lstlisting}
\end{frame}


\begin{frame}[t,fragile]{Invocación a funciones redefinidas}
\begin{lstlisting}
void imprime(const geom::rectangulo & r) {
  std::cout << "centro: " << r.centro() << "\n"; // rectagulo::centro()
}

void imprime(const geom::figura & f) {
  std::cout << "centro: " << f.centro() << "\n"; // figura::centro()
}

int main() {
  using namespace geom;
  rectangulo r{punto{1.5, 1.5}, punto{3.0,3.0}};
  figura & f = r;
  imprime(r);
  imprime(f);
}
\end{lstlisting}

\mode<presentation>{\vfill\pause}
\begin{lstlisting}[style=terminal]
centro: [2.25 , 2.25]
centro: [2.25 , 2.25]
\end{lstlisting}

\end{frame}

\begin{frame}[t,fragile]{Pérdida del tipo dinámico}
\begin{itemize}
  \item La \textmark{vinculación tardía} se consigue a través de 
        punteros y referencias a objetos.
\begin{lstlisting}
  rectangulo r{punto{1.5, 1.5}, punto{3.0,3.0}};
  figura & f = r;
  std::cout << f.centro(); // Vinculación tardía
\end{lstlisting}

  \mode<presentation>{\vfill\pause}
  \item Si se \textmark{copia} un \textgood{objeto derivado} en un \textgood{objeto base}
        se copia \textbad{solamente} la parte del objeto \textgood{base}.
    \begin{itemize}
      \item Se produce un \textmark{recorte} (\emph{slicing}).
      \item Se \textbad{pierde} la información sobre el tipo dinámico.
      \item Toda la invocaciones son de \textmark{vinculación temprana}
    \end{itemize}
\begin{lstlisting}
  rectangulo r{punto{1.5, 1.5}, punto{3.0,3.0}};
  figura f = r; // Cuidado. Recorte
  std::cout << f.centro(); // Vinculación temprana
\end{lstlisting}
\end{itemize}
\end{frame}

\begin{frame}[t,fragile]{Virtualizando funciones no miembro}
\begin{itemize}
  \item El mecanismo de \textgood{funciones virtuales} solamente se puede
        aplicar a funcione miembro.
    \begin{itemize}
      \item El operador \cppkey{<{}<} es una función global.
      \item \textbad{Tiene que ser} una \textgood{función virtual} porque
            el operando izquierdo es un \cppid{std::ostream}.
    \end{itemize}
\begin{lstlisting}
geom::circulo c{punto{1.5,1.5}, 3.0};
std::ofstream archivo{"grafico.txt"};
archivo << c; // operator<<(archivo,c);
\end{lstlisting}

  \mode<presentation>{\vfill\pause}
  \item \textmark{Alternativa}:
    \begin{enumerate}
      \item Definir una \textmark{función miembro virtual} \cppid{inserta()}
            que toma como argumento un \cppid{std::ostream}.
      \item Redefinir \cppid{inserta()} en cada clase de la jerarquía.
      \item Definir el operador \cppkey{<{}<} \textbad{solamente} para la clase base 
            invocando a la función virtual.
    \end{enumerate}

\end{itemize}
\end{frame}

\begin{frame}[t,fragile]
\begin{block}{figura.hpp}
\begin{lstlisting}
class figura {
public:
  //...
  virtual void inserta(std::ostream & os) const;
  //...
};

std::ostream & operator<<(std::ostream & os, const figura & f);
\end{lstlisting}
\end{block}

\mode<presentation>{\vfill\pause}
\begin{block}{figura.cpp}
\begin{lstlisting}
void figura::inserta(std::ostream & os) const {
  os << "figura: " << posicion_;
}

std::ostream & operator<<(std::ostream & os, const figura & f) {
  f.inserta(os); // Vinculación tardía
  return os;
}
\end{lstlisting}
\end{block}
\end{frame}

\begin{frame}[t,fragile]
\begin{block}{rectangulo.hpp}
\begin{lstlisting}
class rectangulo : public figura {
public:
  //...
  void inserta(std::ostream & os) const override;
  //...  
};
\end{lstlisting}
\end{block}

\mode<presentation>{\vfill\pause}
\begin{block}{rectangulo.cpp}
\begin{lstlisting}
void rectangulo::inserta(std::ostream & os) const {
  os << "rectangulo: [";
  figura::inserta(os); // Vinculación temprana
  os << "] , " << ancho() << " , " << alto() << " ]";
}
\end{lstlisting}
\end{block}
\end{frame}

\subsection{Funciones virtuales puras}

\begin{frame}[t,fragile]{Funciones virtuales y clase base} 
\begin{itemize}
  \item Hay operaciones para las que \textbad{no hay una implementación evidente}
        de una función virtual en la clase base.
\begin{lstlisting}[escapechar=@]
class figura {
public:
  [[nodiscard]] virtual double area() const noexcept { return @\color{red}???@; }
  //...
};
\end{lstlisting}

  \mode<presentation>{\vfill\pause}
  \item \textmark{Alternativas}:
    \begin{itemize}
      \item Lanzar una \textbad{excepción }.
\begin{lstlisting}
{ throw logic_error{"area() no implementado}; }
\end{lstlisting}

      \mode<presentation>{\vfill\pause}
      \item Tratarlo como un \textbad{error de programación}.
\begin{lstlisting}
{ CONTRACT_ASSERT(false); }
\end{lstlisting}

      \mode<presentation>{\vfill\pause}
      \item Devolver un \textmark{valor por defecto}.
\begin{lstlisting}
{ return 0; }
\end{lstlisting}

    \end{itemize}
\end{itemize}
\end{frame}

\begin{frame}[t,fragile]{Funciones virtuales puras}
\begin{itemize}
  \item \textbad{Problemas} con las alternativas:
    \begin{itemize}
      \item Los errores y excepciones se detectan en tiempo de ejecución.
        \begin{itemize}
          \item Y solamente si se prueba suficientemente.
        \end{itemize}
      \item Un valor por defecto no fuerza a redefinir en todas las subclases.
    \end{itemize}

  \mode<presentation>{\vfill\pause}
  \item Una \textgood{función virtual pura} fuerza a que cada subclase
        redefina dicha función.
\begin{lstlisting}[escapechar=@]
class figura {
public:
  [[nodiscard]] virtual double area() const noexcept = 0;
  //...
};

class triangulo : public figura {
  // Error si no redefine area() y se instancia un triangulo
};
\end{lstlisting}

\end{itemize}
\end{frame}

\begin{frame}[t,fragile]{Clases abstractas}
\begin{itemize}
  \item Una clase que tiene \textmark{al menos} una \textgood{función virtual pura}
        es una \textemph{clase abstracta}.
    \begin{itemize}
      \item \textbad{No se pueden instanciar} objetos de un \textgood{clase abstracta}.
\begin{lstlisting}
figura f{punto{1.0, 2.0}}; // Error clase abstracta
auto p = make_unique<figura>(punto{3.0,3.0}); // Error clase abstracta
\end{lstlisting}
    \end{itemize}

  \mode<presentation>{\vfill\pause}
  \item Si una subclase \textmark{no redefine} alguna \textgood{función virtual pura}
        la subclase sigue siendo una \textbad{clase bastracta}.
\begin{lstlisting}
class figura {
public:
  virtual void dibuja() = 0;
};

class figura_colorada : public figura {
  // Las subclases deben seguir redefiniendo dibuja()
};
\end{lstlisting}
\end{itemize}
\end{frame}

\begin{frame}[t,fragile]{Colecciones heterogéneas}
\begin{itemize}
  \item El \textmark{polimorfismo dinámico} permite el uso de
        \textmark{colecciones heterogéneas}.
    \begin{itemize}
      \item Un contenedor en que distintos elementos son instancias de 
            distintas subclases.
    \end{itemize}
\begin{lstlisting}
void imprime_areas(const std::vector<std::unique_ptr<figura>> & v) {
  for (auto & f : v) {
    std::cout << f << " -- área: " << f.area() "\n";
  }
}
\end{lstlisting}
  
  \mode<presentation>{\vfill\pause}
  \item No se podría definir un \cppid{std::vector<figura>} porque:
    \begin{itemize}
      \item \cppid{figura} es una clase abstracta.
      \item Al insertar elementos en el vector se produciría un \textbad{recorte}.
      \item Las llamadas función no serían polimórficas.
    \end{itemize}
\end{itemize}
\end{frame}

\begin{frame}[t,fragile]{Destrucción de objetos}
\begin{itemize}
  \item Cuando \textbad{acaba} el \textmark{tiempo de vida} de un objeto 
        se invoca a su \textgood{destructor}.
\begin{lstlisting}
void f() {
  circulo c{punto{1.0,1.0}, 5.0};
  if (cond) {
    retctangulo r{punto{-1.0,-1.0}, punto{1.0,1.0}};
    //...
  } // Invoca a r.rectangulo::~rectangulo()
  //...
} // Invoca a c.circulo::~circulo()
\end{lstlisting}

  \mode<presentation>{\vfill\pause}
  \item Objetos con almacenamiento 
        \textemph{estático} (variables globales) y
        \textemph{automático} (variables locales).
    \begin{itemize}
      \item Se selecciona el \textgood{destructor} 
            definido por el \textmark{tipo estático} del objeto.
    \end{itemize}
\end{itemize}
\end{frame}

\begin{frame}[t,fragile]{Destrucción y polimorfismo}
\begin{itemize}
  \item El \textmark{tiempo de vida} de un objeto alojado 
        en \textmark{memoria dinámica} \textbad{acaba}:
    \begin{itemize}
      \item \textgood{Punteros elegantes}: Al destruir el \textmark{puntero elegante}.
\begin{lstlisting}
auto p = std::make_shared<circulo>(punto{1.0,1.0}, 5.0};
p = nullptr; // Invoca a p->circulo::~circulo() si no hay más refs
\end{lstlisting}
      \mode<presentation>{\vfill\pause}
      \item \textgood{Punteros primitivos}: Al utilizar el operador \cppkey{delete}.
\begin{lstlisting}
circulo * p = new circulo{punto{1.0,1.0}, 5.0};
delete p; // Invoca a p->circulo::~circulo()
\end{lstlisting}
    \end{itemize}

  \mode<presentation>{\vfill\pause}
  \item Cuando se invoca al \textgood{destructor} de un objeto 
        a través de un puntero a la \textgood{clase base}:
    \begin{itemize}
      \item Si el \textmark{destructor} \textbad{no es virtual} $\Rightarrow$ 
            se invoca al destructor de la \textgood{clase base} (tipo estático).
      \item Si el \textmark{destructor} \textgood{es virtual} $\Rightarrow$ 
            se invoca al destructor de la \textgood{clase derivada} (tipo dinámico).
    \end{itemize}
\end{itemize}
\end{frame}

\begin{frame}[t,fragile]{Destructor virtual}
\begin{columns}[T]

\column{.5\textwidth}
\begin{block}{Destructor no virtual}
\begin{lstlisting}
class figura {
public:
  virtual double area() const noexcept;
  ~figura() noexcept;
  //...
};

//...
std::unique_ptr<figura> p = 
    std::make_unique<circulo>(
        punto{1.0,1.0},2.0);
//...
p.reset(); // p->figura::~figura()
\end{lstlisting}
\end{block}

\column{.5\textwidth}
\begin{block}{Efecto del destructor virtual}
\end{block}
\begin{lstlisting}[escapechar=@]
class figura {
public:
  virtual double area() const noexcept;
  @\color{red}virtual@ ~figura() noexcept;
  //...
};

//...
std::unique_ptr<figura> p = 
    std::make_unique<circulo>(
        punto{1.0,1.0},2.0);
//...
p.reset(); // p->circulo::~circulo()
\end{lstlisting}

\end{columns}
\end{frame}


\begin{frame}[t,fragile]{Destructor virtual}
\begin{itemize}
  \item Si una clase tiene \textmark{al menos} una \textgood{función virtual}
        se \textmark{recomienda} hacer su \textemph{destructor virtual}.

  \mode<presentation>{\vfill\pause}
  \item El \textgood{destructor} generado por defecto \textbad{no es} \textemph{virtual}.
\begin{lstlisting}
class figura {
public:
  virtual double area() const noexcept;
  // ~figura() noexcept {} <- Generado por defecto
  virtual ~figura() noexcept {}
  //...
\end{lstlisting}

  \mode<presentation>{\vfill\pause}
  \item Si no se va hacer nada en el \textmark{cuerpo del destructor},
        se puede \textemph{declarar el destructor} con \textgood{implementación por defecto}.
\begin{lstlisting}
class figura {
public:
  //...
  virtual ~figura() noexcept = default;
  //...

\end{lstlisting}

\end{itemize}
\end{frame}


\subsection{Funciones virtuales finales}

\begin{frame}[t,fragile]{Funciones finales}
\begin{itemize}
  \item Una \textemph{función virtual} se puede marcar en una clase
        como una función \textemph{final}.
    \begin{itemize}
      \item Indica que \textbad{ninguna subclase} puede \textmark{redefinir} la función.
    \end{itemize}

\mode<presentation>{\vfill\pause}
\begin{lstlisting}
class figura {
public:
  [[nodiscard]] virtual bool es_simple() const = 0;
  //...
};
class figura_simple : public figura {
public:
  [[nodiscard]] virtual bool es_simple() const final { return true; }
  //...
};
class circulo : public figura_simple {
public:
  [[nodiscard]] virtual bool es_simple() const { return false; } // Error
  //...
};
\end{lstlisting}
\end{itemize}
\end{frame}

\begin{frame}[t,fragile]{Clases finales}
\begin{itemize}
  \item Una \textgood{clase} se puede marcar como \cppkey{final}.
    \begin{itemize}
      \item Convierte todas sus \textgood{funciones miembro} en \textemph{finales}.
    \end{itemize}
\begin{lstlisting}
class circulo final : public figura_simple { /*...*/ };
\end{lstlisting}

  \mode<presentation>{\vfill\pause}
  \item \textmark{Efecto adicional}: \textbad{No se puede heredar} de una \textemph{clase final}.
\begin{lstlisting}
class circulo_coloreado : public circulo { // Error: circulo es final
//...
};
\end{lstlisting}

  \mode<presentation>{\vfill\pause}
  \item \textmark{Impacto}:
    \begin{itemize}
      \item Puede permitir algunas \textgood{optimizaciones}.
      \item Establece \textbad{restricciones} de diseño.
    \end{itemize}
\end{itemize}
\end{frame}

\subsection{Tipos de retorno covariantes}

\begin{frame}[t,fragile]{Variaciones en funciones virtuales}
\begin{itemize}
  \item Una \textmark{redefinición} de una \textemph{función virtual} debe utilizar 
        exactamente los \textmark{mismos tipos} para los \textgood{parámetros}.
    \begin{itemize}
      \item Pero puede haber variaciones en el tipo de retorno.
    \end{itemize}

  \mode<presentation>{\vfill\pause}
  \item Si el tipo de retorno en la función base es \cppid{T\&} o \cppid{T*}:
    \begin{itemize}
      \item El tipo de retorno en la función redefinida puede ser \cppid{D\&} o \cppid{D*}.
      \item \cppid{D} es una clase derivada públicamente de \cppid{T}.
    \end{itemize}

  \mode<presentation>{\vfill\pause}
  \item Si el tipo de retorno en la función base es \cppid{T}:
    \begin{itemize}
      \item El tipo de retorno en la función redefinida debe ser \cppid{T}.
    \end{itemize}
\end{itemize}
\end{frame}

\begin{frame}[t,fragile]{Tipos covariantes y copia}
\begin{itemize}
  \item Un \textgood{constructor} \textbad{no puede ser} \textemph{virtual}.
    \begin{itemize}
      \item Pero se puede utilizar una \textemph{función virtual}
            para crear copias.
    \end{itemize}
\begin{lstlisting}
class figura {
public:
  [[nodiscard]] virtual figura * clona() const = 0;
  //... 
};

class circulo {
public:
  [[nodiscard]] circulo * clona() const override {
    return new circulo{*this};
  }
  //...
};
//...
auto p = f->clona(); // p es figura*
\end{lstlisting}
\end{itemize}
\end{frame}


\begin{frame}[t,fragile]{Tipos covariantes y punteros elegantes}
\begin{itemize}
  \item La regla de los \textgood{tipos covariantes} \textmark{parece forzar}
        al uso de \textemph{punteros primitivos}.

\begin{lstlisting}
class figura {
public:
  [[nodiscard]] std::unique_ptr<figura> clona() const {
    return std::unique_ptr<figura>{clona_impl()};
  }
private:
  [[nodiscard]] virtual figura * clona_impl() const = 0;
};

class circulo {
private:
  [[nodiscard]] circulo * clona_impl() const override {
    return new circulo{*this};
  }
};
//...
auto p = f->clona(); // p es std::unique_ptr<figura>
\end{lstlisting}
\end{itemize}
\end{frame}
