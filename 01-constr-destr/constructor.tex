\section{Constructores}

\subsection{Tipos sin constructor}

\begin{frame}[t,fragile]{Iniciación sin constructor}
\mode<presentation>{\vspace{-1em}}
\begin{itemize}
  \item Iniciación de tipos primitivos.
\begin{lstlisting}
int x{1024}; // int x = 1024
double * p{nullptr}; // double * p = nullptr
\end{lstlisting}

  \mode<presentation>{\vfill\pause}
  \item Clases sin miembros privados que no tienen constructor:
\begin{lstlisting}
struct dispositivo {
  string id_serie;
  long long capacidad;
};
\end{lstlisting}

    \mode<presentation>{\vfill\pause}
    \begin{itemize}
      \item Iniciación por miembros:
\begin{lstlisting}
dispositivo d1{"hda1", 1024};
\end{lstlisting}

      \mode<presentation>{\vfill\pause}
      \item Iniciación directa
\begin{lstlisting}
dispositivo d3{d1}; // Copia miembro a miembro
\end{lstlisting}

      \mode<presentation>{\vfill\pause}
      \item Iniciación por defecto
\begin{lstlisting}
dispositivo d4{};
dispositivo d5;
\end{lstlisting}
    \end{itemize}
\end{itemize}
\end{frame}

\begin{frame}[t,fragile]{Iniciación por defecto}
\begin{itemize}
    \item Iniciación de \textgood{variables globales} $\Rightarrow$ 
          iniciación por defecto de \textmark{todos} los miembros.
\begin{lstlisting}
dispositivo d1; // id_serie="", capacidad=0
dispositivo d2{}; // id_serie="", capaciad=0
\end{lstlisting}

    \mode<presentation>{\vfill\pause}
    \item Iniciación de \textgood{variables locales} $\Rightarrow$ 
          iniciación por defecto \textmark{sólo} de miembros de tipo clase.
        \begin{itemize}
          \item Miembros de tipos primitivos \textbad{sin iniciar}.
        \end{itemize}
\begin{lstlisting}
void f() {
  dispositivo d3; // id_serie="", capacidad=?
  dispositivo d4{}; // id_sere="", capacidad=0
}
\end{lstlisting}
\end{itemize}
\end{frame}

\begin{frame}[t,fragile]{Iniciación de miembros por defecto}
\begin{itemize}  
  \item Se puede indicar valor por defecto para miembros.
\begin{lstlisting}
struct dispositivo {
  string id_serie{"sda1"};
  long long capacidad = 1024; // Notación alternativa
};

dispositivo d; // d.serie=="sda1", d.capacidad==1024
dispositivo e{"sda9"}; // e.serie="sda9", e.capacidad=1024
dispositivo f{2048}; // No compila
\end{lstlisting}

  \mode<presentation>{\vfill\pause}
  \item \textbad{Nota}: También conocida como \textmark{NSDMI}
        (non static data member initializers).
\end{itemize}
\end{frame}

\subsection{Tipos con constructor}

\begin{frame}[fragile]{Invocación al constructor}
\begin{itemize}
  \item Sin un tipo tiene uno o más constructores, 
        todas las definiciones de objetos \textgood{deben invocar} 
        \textmark{algún constructor}.
    \begin{itemize}
      \item El compilador \textbad{deja de generar} un 
            \textmark{constructor por defecto}.
    \end{itemize}
\begin{lstlisting}
struct complejo {
  double real, imag;
  complejo(double r);
  complejo(double r, double i);
};

complejo c1; // Error -> Sin constructor por defecto
complejo c2{}; // Error -> Sin constructor por defecto
complejo c3{2.0}; // OK
complejo c4{2.0, 3.5}; // OK
complejo * pc = new complejo{2.0,3.5}; // OK
complejo v[] { {1.0,1.5}, {2.0, 2.0} }; // OK
std::vector<complejo> w { {1.0, 1.5}, {2.0, 2.0} }; // OK
\end{lstlisting}
\end{itemize}
\end{frame}

\begin{frame}[fragile]{Invocación forzosa de constructor}
\begin{itemize}
  \item Existe una sintaxis para forzar la invocación de un constructor:
        \textmark{uso de paréntesis}.
    \begin{itemize}
      \item No invoca iniciación miembro a miembro o basadas en listas.
    \end{itemize}

\mode<presentation>{\vfill\pause}
\begin{lstlisting}
struct punto {
  double x, y;
};
punto p1(1.0, 1.0); // Error: No hay constructor
punto p2{1.0, 1.0}; // OK
punto p3(1.0); // Error: No hay constructor
punto p4{1.0}; // x=1.0, y=0.0
complejo c1(1.0, 1.0); // OK
complejo c2{1.0, 1.0}; // OK
complejo c3(1.0); // OK
complejo c4{1.0}; // OK
\end{lstlisting}

  \mode<presentation>{\vfill\pause}
  \item Útil para discriminar entre constructores de \cppid{std::vector} de tipos numéricos.
\begin{lstlisting}
std::vector<int> v{10}; // Vector con el valor 10. v.size() == 1
std::vector<int> v(10); // Vector con 10 valores 0. v.size() == 10
\end{lstlisting}
\end{itemize}
\end{frame}

\subsection{Constructor por defecto}

\begin{frame}[t,fragile]{Constructor por defecto}
\begin{itemize}
  \item El \textgood{constructor vacío} o \textmark{constructor por defecto} 
        es un constructor que no toma ningún parámetro.
\begin{lstlisting}
vecnum();
\end{lstlisting}

  \mode<presentation>{\vfill\pause}
  \item Define que debe hacerse:
    \begin{itemize}
      \item Cuando se \textmark{define un objeto} sin valor inicial.
\begin{lstlisting}
vecnum v;
vecnum v{};
\end{lstlisting}

      \mode<presentation>{\pause}
      \item Cuando se \textmark{asigna memoria} sin valor inicial
\begin{lstlisting}
vecnum * pv = new vecnum;
\end{lstlisting}

    \end{itemize}
\end{itemize}
\end{frame}

\begin{frame}[t,fragile]{Generación del constructor por defecto}
\mode<presentation>{\vspace{-.5em}}
\begin{itemize}
  \item El \textgood{constructor por defecto} se genera 
        \textbad{si no hay ningún constructor}.

\mode<presentation>{\vspace{-.75em}}
\begin{columns}[T]
\column{.2\textwidth}
\column{.5\textwidth}
\begin{lstlisting}
class vecnum {
public:
  //...
  //...
};
vecnum v; // OK
\end{lstlisting}

\column{.5\textwidth}
\begin{lstlisting}
class vecnum {
public:
  vecnum(int n);
  //...
};
vecnum v; // Error
\end{lstlisting}

\end{columns}

\mode<presentation>{\vfill\pause}
\item Se puede \textbad{forzar} la generación del \textgood{constructor por defecto} o
      \textbad{inhibir} su generación.

\mode<presentation>{\vspace{-.75em}}
\begin{columns}[T]
\column{.2\textwidth}
\column{.5\textwidth}
\begin{lstlisting}
class vecnum {
public:
  vecnum() = default;
  vecnum(int n);
  //...
};
vecnum v; // OK
\end{lstlisting}

\column{.5\textwidth}
\begin{lstlisting}
class vecnum {
public:
  vecnum() = delete;
  //...
  //...
};
vecnum v; // Error
\end{lstlisting}

\end{columns}

\end{itemize}
\end{frame}

\begin{frame}[t,fragile]{Constructor por defecto y arrays}
\begin{itemize}
  \item El tipo base de un array debe tener \textgood{constructor por defecto}, 
        sólo si hace falta iniciar por defecto.
\begin{lstlisting}
struct punto {
  double x, y;
  punto();
};
struct complejo {
  double real, imag;
  complejo(double d, double i);
};

std::array<punto,16> v; // OK
std::array<complejo,16> w; // Error: No pueden iniciarse elementos
std::array<complejo,2> c { {1.0, 1.0}, {2.0, 2.0} }; // OK 
\end{lstlisting}
\end{itemize}
\end{frame}

\begin{frame}[t,fragile]{Constructor por defecto y vectores}
\begin{itemize}
  \item El tipo base de un vector debe tener \textgood{constructor por defecto}, 
        si hace falta iniciar por defecto.
\begin{lstlisting}
struct punto {
  double x, y;
  punto();
};
struct complejo {
  double real, imag;
  complejo(double d, double i);
};

std::vector<punto> vp1; // OK. No necesita iniciar
std::vector<complejo> vc1; // OK. No necesita iniciar

std::vector<punto> vp2(16); // OK. 16 elementos con valor por defecto
std::vector<complejo> vc2(16); // Error: no puede iniciar 16 elementos
std::vector<complejo> vc3{ {1.0, 1.0}, {2.0, 2.0} }; // OK
std::vector<complejo> vc4{5, {1.0,1.0}}; // OK: 5 elementos iniciados
\end{lstlisting}
\end{itemize}
\end{frame}

\subsection{Conversión y constructor explícito}

\begin{frame}[fragile]{Constructor explícito}
\begin{itemize}
  \item Un constructor que recibe un único argumento define una 
        \textmark{operación de conversión}.
\begin{lstlisting}
class complejo {
  complejo(double r, double i);
  complejo(dobule r);
  // ...
};
complejo c = 3.5; // Conversión a complejo
\end{lstlisting}

  \mode<presentation>{\vfill\pause}
  \item Sin embargo:
\begin{lstlisting}
vecnum v = 5; // Conversión de entero a vecnum. Confuso.
\end{lstlisting}

  \mode<presentation>{\vfill\pause}
  \item Se puede forzar a que un constructor no actúe como operación de conversión:
\begin{lstlisting}
explicit vecnum(int n);
\end{lstlisting}

  \mode<presentation>{\vfill\pause}
  \item \textbad{Recomendación}: Marca como explícitos todos los constructores
        de un único parámetro.
\end{itemize}
\end{frame}

\subsection{Constructor basado en lista}

\begin{frame}[fragile]{Lista de iniciación}
\begin{itemize}
  \item Un \textgood{constructor basado en lista} es un constructor que toma 
        \textmark{un único argumento} de tipo \cppid{std::initializer\_list}.
    \begin{itemize}
      \item Permiten definir la construcción de objetos a partir de una 
            \textmark{lista homogénea} de valores.
    \end{itemize}
\begin{lstlisting}
lista l { 1, 2, 3, 4 }; // Invoca a lista(initializer_list<int>)
\end{lstlisting}

  \mode<presentation>{\vfill\pause}
  \item Cualquier función puede tomar como parámetro una lista de iniciación.
    \begin{itemize}
      \item El argumento debe pasarse por valor.
    \end{itemize}
\begin{lstlisting}
int maximo(initializer_list<int> l);

void f() {
  x = maximo({1, 3, 4, 2});
  // ...
}
\end{lstlisting}
\end{itemize}
\end{frame}

\begin{frame}[fragile]{\textbf{initializer\_list}}
\begin{itemize}
  \item La clase \cppid{std::initializer\_list} ofrece funciones miembro 
        para acceder a los valores de la lista:
    \begin{itemize}
      \item \cppid{begin()}: Iterador/puntero al principio de la lista.
      \item \cppid{end()}: Iterador/puntero al final de la lista (siguiente del último).
      \item \cppid{size()}: Tamaño de la lista.
    \end{itemize}

  \mode<presentation>{\vfill\pause}
  \item Son suficientes para recorrer la lista:
\begin{lstlisting}
void imprime(std::initializer_list<int> l) {
  for (auto p=l.begin(); p!=l.end(); ++p) {
    cout << * p << '\n';
  }
}
\end{lstlisting}

  \mode<presentation>{\vfill\pause}
  \item O mejor:
\begin{lstlisting}
void imprime(std::initializer_list<int> l) {
  for (auto x : l) {
    cout << x << endl;
  }
}
\end{lstlisting}
\end{itemize}
\end{frame}

\subsection{Constructores para \textbf{vecnum}}

\begin{frame}[t,fragile]{Constructores}
\begin{itemize}
  \item Constructor vacío
  \item Constructor con tamaño pasa a ser explícito.
  \item Constructor con lista de iniciación.
\end{itemize}
\begin{block}{vecnum.hpp}
\begin{lstlisting}
class vecnum {
public:
  vecnum() : tamanyo_{0}, buffer_{nullptr} {}
  explicit vecnum(int n) : tamanyo_{n}, buffer_{new double[n]{}} {}
  vecnum(std::initializer_list<double> l);
  ~vecnum() { delete []buffer_; }
  //...
\end{lstlisting}
\end{block}
\end{frame}

\begin{frame}[t,fragile]{Implementando el constructor de lista}
\begin{block}{vector.cpp}
\lstinputlisting{ejemplos/01-constr-destr/primitive-vecnum-completo/vecnum.cpp}
\end{block}
\end{frame}
