\section{Garantías de excepciones}

\begin{frame}[t]{Invariantes y estado válido}
\begin{itemize}
  \item \textgood{Invariantes de clase}: Condiciones que comple un objeto
        establecidas por el \textmark{constructor} y mantenidas por todas las
        \textmark{operaciones} hasta la ejecución del \textmark{destructor}.

  \mode<presentation>{\vfill\pause}
  \item Tipo \cppid{vecnum}:
    \begin{itemize}
      \item \cppid{tamanyo\_ >= 0}.
      \item Si \cppid{tamanyo\_ == 0} $\Rightarrow$ \cppid{buffer\_ ==}\cppkey{nullptr}.
      \item Si \cppid{tamanyo\_ > 0} $\Rightarrow$ \cppid{buffer\_} apunta
            a un bloque de memoria dinámica capaz de albergar desde \cppid{buffer\_[0])}
            a \cppid{buffer\_[tamanyo\_-1]}.
    \end{itemize}

  \mode<presentation>{\vfill\pause}
  \item Un \textmark{objeto} se encuentra en un \textgood{estado válido} si:
    \begin{enumerate}
      \item Su constructor se ha ejecutado.
      \item Todas las operaciones ejecutadas mantienen las invariantes.
      \item Su destructor no se ha ejecutado.
    \end{enumerate}
\end{itemize}
\end{frame}

\begin{frame}[t]{Seguridad frente a excepciones}
\begin{itemize}
  \item Una \textmark{operación} es \textemph{segura frente a excepciones}
        (\emph{exception safe}) si la operación deja al programa en un 
        \textgood{estado válido} incluso cuando la operación 
        \textbad{termina lanzando una excepción}.

  \mode<presentation>{\vfill\pause}
  \item Cada operación debe ofrecer la \textemph{garantía mas estricta posible}
        de las siguientes:
    \begin{itemize}

      \mode<presentation>{\vfill\pause}
      \item \textgood{Garantía \emph{nothrow}} $\Rightarrow$ Algunas operaciones.
        \begin{itemize}
          \item La operación no lanza excepciones.
        \end{itemize}
      \mode<presentation>{\vfill\pause}

      \item \textgood{Garantía fuerte} $\Rightarrow$ Operaciones clave.
        \begin{itemize}
          \item La operación se completa o se deshacen los cambios.
        \end{itemize}

      \mode<presentation>{\vfill\pause}
      \item \textgood{Garantía básica} $\Rightarrow$ Todas las operaciones.
        \begin{itemize}
          \item Si hay error, el objeto puede seguir usándose.
        \end{itemize}
    \end{itemize}
\end{itemize}
\end{frame}

\begin{frame}[t,fragile]{Garantía básica}
\begin{itemize}
  \item La \textgood{garantía básica} requiere:
    \begin{itemize}
      \item Se mantienen las invariantes de todos los objetos.
      \item No hay goteo (pérdida) de ningún recurso.
    \end{itemize}

  \mode<presentation>{\vfill\pause}
  \item \textmark{Implicaciones}:
    \begin{itemize}
      \item El objeto puede destruirse.
      \item Se puede asignar otro valor a un objeto.
    \end{itemize}
\end{itemize}

\mode<presentation>{\vfill\pause}
\begin{lstlisting}
vecracional & vecracional::operator=(const vecracional & v) {
  tamanyo_ = v.tamanyo_;
  delete []buffer_;
  buffer_ = new racional[tamanyo_]; // bad_alloc, racional()
  for (int i=0; i<tamanyo_; ++i) {
    buffer_[i] = v.buffer_[i]; // operator=
  }
  return *this;
}
\end{lstlisting}
\end{frame}

\begin{frame}[t,fragile]{Cumpliendo la garantía básica}
\begin{itemize}
  \item \textmark{Razones} para fallo de la sentencia:
\begin{lstlisting}
buffer_ = new racional[tamanyo_];
\end{lstlisting}
    \begin{itemize}
      \item Excepción \cppid{std::bad\_alloc}: No queda memoria.
      \item Alguna excepción lanzada por constructor de \cppid{racional}.
    \end{itemize}

  \mode<presentation>{\vfill\pause}
  \item \textmark{Efectos}:
    \begin{itemize}
      \item \cppid{buffer\_} quedaría apuntando a memoria liberada.
      \item \textbad{No se puede} ejecutar el \textgood{destructor}.
    \end{itemize}
\end{itemize}

\mode<presentation>{\vfill\pause}
\begin{lstlisting}
vecracional & vecracional::operator=(const vecracional & v) {
  auto aux = new racional[v.tamanyo_]; // bad_alloc, racional()
  tamanyo_ = v.tamanyo_;
  delete []buffer_;
  buffer_ = aux;
  for (int i=0; i<tamanyo_; ++i) {
    buffer_[i] = v.buffer_[i]; // operator=
  }
  return *this;
}
\end{lstlisting}
\end{frame}

\begin{frame}[t,fragile]{Garantía fuerte}
\begin{itemize}
  \item La \textmark{garantía fuerte} requiere:
    \begin{itemize}
      \item Cumplir la garantía básica, y además,
      \item O la operación tiene éxito o no tiene efecto.
    \end{itemize}

  \mode<presentation>{\vfill\pause}
  \item \textmark{Consecuencias}:
    \begin{itemize}
      \item Ejecutar todas las operaciones que podrían fallar antes de empezar 
            a modificar el estado del objeto.
    \end{itemize}
\end{itemize}

\mode<presentation>{\vfill\pause}
\begin{lstlisting}
vecracional & vecracional::operator=(const vecracional & v) {
  auto aux = new racional[v.tamanyo_]; // bad_alloc, racional()
  std::copy_n(v.buffer_, v.tamanyo_, aux); // operator=
  tamanyo_ = v.tamanyo_;
  delete []buffer_;
  buffer_ = aux;
  return *this;
}
\end{lstlisting}
\end{frame}

\begin{frame}[t,fragile]{Garantía \textbf{nothrow}}
\begin{itemize}
  \item La \textmark{garantía \emph{nothrow}} requiere:
    \begin{itemize}
      \item Cumplir la garantía fuerte, y además,
      \item No lanzar ninguna excepción.
    \end{itemize}
\end{itemize}

\mode<presentation>{\vfill\pause}
\begin{lstlisting}[basicstyle=\tiny,escapechar=@]
class vecracional {
public:
  //...
  const & racional operator[](int i) const { // Garantía nothrow
    return buffer_[i];
  }

  racional & operator[](int i) { // Garantía nothrow
    return buffer_[i];
  }
  @\pause@
  const racional & posicion(int i) const { // Garantía fuerte
    if (i<0 || i>=tamanyo_) throw std::range_error{"racional::posicion()"};
    return buffer_[i];
  }

  racional & posicion(int i) { // Garantía fuerte
    if (i<0 || i>=tamanyo_) throw std::range_error{"racional::posicion()"};
    return buffer_[i];
  }
  //...
};
\end{lstlisting}
\end{frame}
