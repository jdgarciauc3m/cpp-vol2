\section{Especificiaciones de excepciones}

\begin{frame}[t,fragile]{Funciones que lanzan excepciones}
\begin{itemize}
  \item Normalmente, una función, puede lanzar una excepción.
\begin{lstlisting}
void suma(int x, int y); // Puede lanzar excepciones
\end{lstlisting}

  \mode<presentation>{\vfill\pause}
  \item Se puede utilizar la especificación \cppkey{noexcept} para
        indicar que una función \textmark{ofrece la garantía \emph{noexcept}}.
\begin{lstlisting}
void suma(int x, int y) noexcept; // No lanza excepciones
\end{lstlisting}

  \mode<presentation>{\vfill\pause}
  \item La claúsula \cppkey{noexcept} acepta un argument booleano.
\begin{lstlisting}
void suma(int x, int y) noexcept(true); // No lanza excepciones
void suma(int x, int y) noexcept(false); // Puede lanzar excepciones
\end{lstlisting}


  \mode<presentation>{\vfill\pause}
  \item \textbad{No se comprueba} en tiempo de compilación 
        si la especificación es cierta.
    \begin{itemize}
      \item Requeriría análisis completo de programa durante la fase de enlace.
    \end{itemize}
\end{itemize}
\end{frame}

\begin{frame}[t,fragile]{Violación de especificación de excepciones}
\begin{itemize}
  \item Si una función marcada con \cppkey{noexcept} acaba lanzando una
        excepción, se ha producido una \textbad{violación de especificación noexcept}.

  \mode<presentation>{\vfill\pause}
  \item \textgood{Efecto}:
    \begin{itemize}
      \item El programa \textmark{termina su ejecución} invocando a \cppid{std::terminate()}.
      \item \textbad{No hay garantía} de que se ejecuten los destructores desde el punto
            de \cppkey{throw} al punto de \cppkey{noexcept}.
      \item \textbad{No hay garantía} de 
            \textmark{desenrollado de la pila} (\emph{stack unwinding}).
      \item \textbad{No hay posibilidad} de \textmark{recuperarse} y 
            \textmark{continuar} la ejecución del programa.
    \end{itemize}

  \mode<presentation>{\vfill\pause}
  \item Cuando se marca una función como \cppid{noexcept} se está indicando
        que esta función \textmark{no propaga excepciones}.
\end{itemize}
\end{frame}

\begin{frame}[t,fragile]{Garantía \emph{nothrow}}
\begin{itemize}
  \item La \textmark{garantía \emph{nothrow}} es más difícil de obtener
        de lo que puede parecer.

\mode<presentation>{\vfill\pause}
\begin{lstlisting}[escapechar=@]
void cubo(int x) noexcept {
  return x*x*x;
}

std::vector<int> cubo(const std::vector<int> & v) @{\color{red}noexcept@ {
  std::vector<int> resultado(v.size()); // Podría lanzar std::bad_alloc
  for (std::size_t i=0; i<v.size(); ++i) {
    r.at(i) = cubo(v.at(i)); // Podría lanzar std::out_of_range?
  }
  return r;
}
\end{lstlisting}

  \mode<presentation>{\vfill\pause}
  \item Cualquier operación que implique reserva de memoria podría 
        \textmark{potencialmente} \textbad{lanzar una excepción}
        \cppid{std::bad\_alloc}.

  \mode<presentation>{\vfill\pause}
  \item A veces se puede deducir del contexto que una función 
        \textgood{no lanzará excepciones}.

\end{itemize}
\end{frame}

\begin{frame}[t,fragile]{Operador \emph{noexcept}}
\begin{itemize}
  \item La palabra reservada \cppkey{noexcept} se puede usar como un operador
        sobre una expresión.
    \begin{itemize}
      \item Se puede aplicar a cualquier expresión.
      \item Produce un resultado \textemph{booleano}.
      \item Indica si la evaluación de la expresión es \cppkey{noexcept}.
    \end{itemize}
\begin{lstlisting}
constexpr bool cond = noexcept(v.at(0));
\end{lstlisting}

  \mode<presentation>{\vfill\pause}
  \item Se puede usar para expresar la condición \cppkey{noexcept} de una función.
\begin{lstlisting}
void f(std::vector<double> & v) noexcept(noexcept(v.at(0)));
\end{lstlisting}

  \mode<presentation>{\vfill\pause}
  \item \textmark{Evaluación}:
    \begin{itemize}
      \item La expresión se evalúa en tiempo de compilación solamente.
      \item Se comprueba si las subexpresiones están marcadas como \cppkey{noexcept}.
    \end{itemize}

\end{itemize}
\end{frame}

\begin{frame}[t,fragile]{Excepciones y destructores}
\begin{itemize}
  \item En general, un destructor \textbad{no debería lanzar excepciones}.
    \begin{itemize}
      \item Los destructores de la biblioteca estándar son \cppkey{noexcept}.
      \item El destructor para un tipo es, por defecto, \cppkey{noexcept}
            si todos los miembros del tipo tienen destructores \cppkey{noexcept}.
    \end{itemize}

\mode<presentation>{\vfill}
\begin{lstlisting}
class estudiante {
public:
  //...
  ~estudiante(); // Implícitamente noexcept
private:
  std::string nombre;
  std::vector<std::string> calificaciones;
};
\end{lstlisting}
\end{itemize}
\end{frame}
