\section{Especificiaciones de excepciones}

\begin{frame}[t,fragile]{Funciones que lanzan excepciones}
\begin{itemize}
  \item Normalmente, una función, puede lanzar una excepción.
\begin{lstlisting}
void suma(int x, int y); // Puede lanzar excepciones
\end{lstlisting}

  \mode<presentation>{\vfill\pause}
  \item Se puede utilizar la especificación \cppkey{noexcept} para
        indicar que una función \textbad{no lanza nunca excepciones}.
\begin{lstlisting}
void suma(int x, int y) noexcept; // No lanza excepciones
\end{lstlisting}

  \mode<presentation>{\vfill\pause}
  \item \textbad{No se comprueba} en tiempo de compilación 
        si la especificación es cierta.
    \begin{itemize}
      \item Requeriría análisis completo de programa durante la fase de enlace.
    \end{itemize}
\end{itemize}
\end{frame}

\begin{frame}[t,fragile]{Violación de especificación de excepciones}
\begin{itemize}
  \item Si una función marcada con \cppkey{noexcept} acaba lanzando una
        excepción, se ha producido una \textbad{violación de especificación noexcept}.

  \mode<presentation>{\vfill\pause}
  \item \textgood{Efecto}:
    \begin{itemize}
      \item El programa \textmark{termina su ejecución} invocando a \cppid{std::terminate()}.
      \item \textbad{No hay garantía} de que se ejecuten los destructores desde el punto
            de \cppkey{throw} al punto de \cppkey{noexcept}.
      \item \textbad{No hay garantía} de 
            \textmark{desenrollado de la pila} (\emph{stack unwinding}).
      \item \textbad{No hay posibilidad} de \textmark{recuperarse} y 
            \textmark{continuar} la ejecución del programa.
    \end{itemize}

  \mode<presentation>{\vfill\pause}
  \item Cuando se marca una función como \cppid{noexcept} se está indicando
        que esta función no propaga excepciones.
\end{itemize}
\end{frame}
