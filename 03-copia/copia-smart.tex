\section{Copia y punteros elegantes}

\begin{frame}[t,fragile]{Punteros elegantes en \textbf{vecnum}}
\begin{itemize}
  \item Se puede usar un \cppid{unique\_ptr} para representar el 
        \textmark{búfer interno} de \cppid{vecnum}.
\begin{block}{vecnum.hpp}
\begin{lstlisting}
class vecnum {
public:
  vecnum() = default;
  explicit vecnum(int n) 
      : tamanyo_{n}, buffer_{std::make_unique<double[]>(n)} {}
  vecnum(std::initializer_list<double> l);
  //...
private:
  int tamanyo_ = 0;
  std::unique_ptr<double[]> buffer_ {};
};
\end{lstlisting}
\end{block}
\end{itemize}
\end{frame}

\begin{frame}[t,fragile]{Constructor de lista con \textbf{unique\_ptr}}
\begin{block}{vecnum.cpp}
\begin{lstlisting}
vecnum::vecnum(std::initializer_list<double> l)
    : tamanyo_ {static_cast<int>(l.size())},
      buffer_ {std::make_unique<double[]>(l.size())}
{
  std::copy_n(l.begin(), l.size(), buffer_.get());
}
\end{lstlisting}
\end{block}

\mode<presentation>{\vfill}
\begin{itemize}
  \item \cppid{buffer\_.get()} devuelve el puntero gestionado por el
        \cppid{unique\_ptr}.
    \begin{itemize}
      \item Permite que \cppid{copy\_n()} escriba en 
            \cppid{buffer\_[0]}, \cppid{buffer\_[1]}, \ldots
    \end{itemize}
\end{itemize}
\end{frame}

\begin{frame}[t,fragile]{Copiando punteros elegantes}
\begin{itemize}
  \item El tipo \cppid{unique\_ptr} tiene \textbad{eliminados}:
    \begin{itemize}
      \item El \textmark{constructor de copia}.
      \item El \textmark{operador de asignación de copia}.
    \end{itemize}
\begin{lstlisting}
auto p = std::make_unique<double>();
auto q = p; // Error. No se puede construir por copia.

std::unique_ptr<double> r{};
r = p; // Error no se puede asignar
\end{lstlisting}

  \mode<presentation>{\vfill\pause}
  \item \textbad{Efecto}: No se generan operaciones de copia para una clase
        que tenga un dato miembro de tipo \cppid{unique\_ptr}.
    \begin{itemize}
      \item Hay que definir las operaciones para la clase.
    \end{itemize}
\end{itemize}
\end{frame}

\begin{frame}[t]{Un \textbf{vecnum} con punteros elegantes}
\begin{block}{vecnum.hpp}
\lstinputlisting[lastline=15]{ejemplos/03-copia/vecnum-smart/vecnum.hpp}
\end{block}
\end{frame}

\begin{frame}[t]
\begin{block}{vecnum.hpp}
\lstinputlisting[firstline=17]{ejemplos/03-copia/vecnum-smart/vecnum.hpp}
\end{block}
\end{frame}

\begin{frame}[t]
\begin{block}{vecnum.cpp}
\lstinputlisting[lastline=14]{ejemplos/03-copia/vecnum-smart/vecnum.cpp}
\end{block}
\end{frame}

\begin{frame}[t]
\begin{block}{vecnum.cpp}
\lstinputlisting[firstline=16]{ejemplos/03-copia/vecnum-smart/vecnum.cpp}
\end{block}
\end{frame}
