\section{Copia frente a movimiento}

\begin{frame}[t,fragile]{Intercambio de valores}
\begin{itemize}
  \item Función de intercambio para enteros.
\begin{lstlisting}
void intercambia(int & x, int & y) {
  int aux = x;
  x = y;
  y = aux;
}
//...
int a = 2, b = 4;
intercambia(a,b); // a==4, b==2
\end{lstlisting}

  \mode<presentation>{\vfill\pause}
  \item Se realizan 3 copias de enteros.
    \begin{enumerate}
      \item Iniciación de \cppid{aux} $\Rightarrow$ 1 copia.
      \item Asignación a \cppid{x}: $\Rightarrow$ 1 copia.
      \item Asignación a \cppid{y}: $\Rightarrow$ 1 copia.
    \end{enumerate}
\end{itemize}
\end{frame}

\begin{frame}[t,fragile]{Intercambio de valores compuetos}
\begin{itemize}
  \item Función de intercambio para vectores numéricos.
\begin{lstlisting}
void intercambia(vecnum & v, vecnum & w) {
  vecnum aux{v};
  v = w;
  w = aux;
}
\end{lstlisting}

  \mode<presentation>{\vfill\pause}
  \item Asignaciones de memoria y copias de valores.
    \begin{enumerate}
      \item Iniciación de \cppid{aux} $\Rightarrow$ 
            1 reserva + 
            1 copia de entero + 1 copia de puntero +
            1 copia de elementos.
      \item Asignación a \cppid{v} $\Rightarrow$ 
            1 liberación + 1 reserva + 
            1 copia de entero + 1 copia de puntero +
            1 copia de elementos.
      \item Asignación a \cppid{w} $\Rightarrow$ 
            1 liberación + 1 reserva + 
            1 copia de entero + 1 copia de puntero +
            1 copia de elementos.
      \item Destrucción de \cppid{aux} $\Rightarrow$ 1 liberación.
    \end{enumerate}

  \mode<presentation>{\vfill\pause}
  \item \textbad{Lejos} de la solución óptima.
\end{itemize}
\end{frame}

\begin{frame}[t,fragile]{Si se tuviese acceso a la implementación}
\begin{itemize}
  \item Función de intercambio para vectores numéricos.
\begin{lstlisting}
void intercambia(vecnum & v, vecnum & w) {
  int aux_tamanyo = v.tamanyo_;
  v.tamanyo_ = w.tamanyo_;
  w.tamanyo_ = aux_tamanyo_;
  double * aux_buffer = v.buffer_;
  v.buffer_ = w.buffer_;
  w.buffer_ = aux_buffer;
}
\end{lstlisting}

  \mode<presentation>{\vfill\pause}
  \item Asignaciones de memoria y copias de valores.
    \begin{enumerate}
      \item Iniciación de \cppid{aux\_tamanyo}: 1 copia.
      \item Asignaciones de \cppid{v.tamanyo\_} y \cppid{w.tamanyo\_}: 2 copias.
      \item Iniciación de \cppid{aux\_buffer}: 1 copia.
      \item Asignaciones de \cppid{v.buffer\_} y \cppid{w.buffer\_}: 2 copias.
    \end{enumerate}

  \mode<presentation>{\vfill\pause}
  \item \textgood{Solución óptima} \textbad{si se pudiese acceder a detalles internos}.
\end{itemize}
\end{frame}

\begin{frame}[t]{Comparación}
\begin{center}
\begin{tabular}[t]{|l|r|r|}
\hline
& \textgood{Sin acceso} & \textgood{Con acceso}\\
\hline
\hline
Reserva memoria & \textbad{3} & 0\\
\hline
Liberación de memoria & \textbad{3} & 0\\
\hline

Copias de enteros & 3 & 3\\
\hline

Copias de punteros & 3 & 3\\
\hline

Copias de elementos & \textbad{3}\footnote{El número elementos depende de \cppid{v} y \cppid{w}} & 0\\
\hline

\end{tabular}
\end{center}

\mode<presentation>{\vfill\pause}
\begin{itemize}
  \item El intercambio eficiente \textbad{no se puede implementar} en términos de copias.

  \mode<presentation>{\vfill\pause}
  \item La función \cppid{intercambia()} es un ejemplo: hay otros casos.
    \begin{itemize}
      \item Es necesaria una \textmark{solución general} al problema.
    \end{itemize}
\end{itemize}

\end{frame}
