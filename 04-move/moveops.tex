\section{Operaciones de movimiento}

\begin{frame}[t,fragile]{Movimiento explícito}
\begin{itemize}
  \item Un tipo de datos en C++ puede soportar \textgood{operciones de movimiento}.
    \begin{itemize}
      \item Muchos tipos de la biblioteca estándar soportan el movimiento.
    \end{itemize}
\begin{lstlisting}
std::string s = "Daniel";
std::string t = "Carlos";
std::string u = "María";
s = t; // Copia -> s=="Carlos", t=="Carlos"
s = std::move(u); // Movimiento -> s=="María", u=???
\end{lstlisting}

  \mode<presentation>{\vfill\pause}
  \item Una \textgood{operación de movimiento} adquiere los recursos del
        origen y los deja en un estado válido pero no especificado.

  \mode<presentation>{\vfill\pause}
  \item \textbad{Recuerda}: La función \cppid{std::move(x)} no mueve el objeto \cppid{x}
        pero lo hace susceptible de ser movido.

\end{itemize}
\end{frame}

\begin{frame}[t,fragile]{Intercambio de objetos movibles}
\begin{itemize}
  \item Se puede implementar el intercambio en función de operaciones de movimiento.
    \begin{itemize}
      \item El tipo debe ser \textmark{movible}.
    \end{itemize}
\end{itemize}
\begin{lstlisting}[escapechar=@]
void intercambia(std::string & s, std::string & t) {
  std::string aux{std::move(s)};
  s = std::move(t);
  t = std::move(aux);
}
@\pause@
void intercambia(std::vector<double> & v, std::vector<double> & w) {
  std::vector<double> aux{std::move(v)};
  v = std::move(w);
  w = std::move(aux);
}
\end{lstlisting}
\end{frame}

\begin{frame}[t,fragile]{Intercambio de objetos no movibles}
\begin{itemize}
  \item Los tipos definidos por el usuario que tienen \textmark{operaciones de copia} 
        definidas son tipos \textbad{no movibles}.
    \begin{itemize}
      \item Las \textmark{operaciones explícitas de movimiento} se transforman en 
            \textgood{operaciones de copia}.
    \end{itemize}
\begin{lstlisting}
void intercambia(vecnum & v, std::vecnum & w) {
  vecnum aux{std::move(v)};
  v = std::move(w);
  w = std::move(aux);
}
\end{lstlisting}

  \mode<presentation>{\vfill\pause}
  \item \textbad{Cuidado}: Todos los movimientos transformados en copias.
    \begin{itemize}
      \item Sigue siendo una operación costosa.
    \end{itemize}
\end{itemize}
\end{frame}

\begin{frame}[t,fragile]{Movimiento y retorno de objetos locales}
\begin{itemize}
  \item Cuando una función devuelve una \textmark{copia de un objeto local}, 
        se usa \textemph{implícitamente} el \textgood{movimiento}.
\begin{lstlisting}
std::string concatena(const std::string & a, const std::string & b) {
  std::string resultado = b + ", " + a;
  return resultado;
}

x = concatena("Daniel", "García"); // x = std::move(resultado);
\end{lstlisting}

  \mode<presentation>{\vfill\pause}
  \item Se aprovecha que \cppid{resultado} se va a \textbad{destruir} 
        al \textmark{salir} de \cppid{concatena()}.

  \mode<presentation>{\vfill\pause}
  \item \textmark{Proceso}:
    \begin{itemize}
      \item \cppid{x} adquire los recursos de \cppid{resultado}.
      \item \cppid{resultado} se destruye.
    \end{itemize}
\end{itemize}
\end{frame}

\begin{frame}[t,fragile]{Elusión de copia/movimiento}
\begin{itemize}
  \item Cuando una función devuelve un \textmark{objeto temporal},
        se \textemph{elude} la operación de \textgood{copia} o \textgood{movimiento}. 
\begin{lstlisting}
std::string concatena(const std::string & a, const std::string & b) {
  return b + ", " + a;
}

x = concatena("Daniel", "García"); // x = b + ", " + a;
\end{lstlisting}

\end{itemize}
\end{frame}

\begin{frame}[t,fragile]{Movimiento y paso por valor}
\begin{itemize}
  \item Una función que toma un parámetro por valor puede invocar
        a una \textmark{operación de copia} o una \textmark{operación de movimiento}.
\begin{lstlisting}
void imprime(std::string s);
//...
std::string n{"Daniel"};
imprime(n); // Copia n
imprime(std::move(n)); // Mueve n
imprime(concatena("Daniel","García")); // Mueve el resultado de concatena()
\end{lstlisting}
\end{itemize}
\end{frame}
