\section{Movimiento en tipos compuestos}

\begin{frame}[t,fragile]{Composición y movimiento}
\mode<presentation>{\vspace{-.75em}}
\begin{block}{Tipos compuestos}
\begin{lstlisting}
class muestra {
public:
  //...
  muestra(muestra && m) : id_{m.id_}, datos_{m.datos_} {}
  muestra & operator=(muestra && m) {
    if (this==&m) return *this;
    id_ = m.id_;
    datos_ = m.datos_;
    return *this;
  }
private:
  std::string id_;
  vecnum datos_;
};
\end{lstlisting}
\end{block}

\begin{itemize}
  \mode<presentation>{\vfill\pause}
  \item \textbad{Problema}: Copia en vez de movimiento.
    \begin{itemize}
      \pause
      \item \cppid{m.id\_} y \cppid{m.datos\_} no son objetos temporales.
    \end{itemize}
\end{itemize}

\end{frame}


\begin{frame}[t,fragile]{Composición y movimiento resueltos}
\begin{block}{Tipos compuestos}
\begin{lstlisting}[escapechar=@]
class muestra {
public:
  //...
  muestra(muestra && m) : id_{@\color{red}std::move@(m.id_)}, datos_{@\color{red}std::move@m.datos_)} {}
  muestra & operator=(muestra && m) {
    if (this == &m) return *this;
    id_ = @\color{red}std::move@(m.id_);
    datos_ = @\color{red}std::move@(m.datos_);
    return *this;
  }
private:
  std::string id_;
  vecnum datos_;
};
\end{lstlisting}
\end{block}

\end{frame}


\begin{frame}[t]{Reglas de generación de funciones especiales}
\begin{itemize}
  \item Si una clase \textbad{no tiene ninguna} de las siguientes operaciones 
        \textmark{declaradas por el usuario}:
    \begin{itemize}
      \item \textgood{Constructor de copia}.
      \item \textgood{Operador de asignación de copia}.
      \item \textgood{Constructor de movimiento}.
      \item \textgood{Operador de asignación de movimiento}.
      \item \textgood{Destructor}.
    \end{itemize}

  \mode<presentation>{\vfill}
  \item \textmark{Entonces}:
    \begin{itemize}
      \item Se genera un \textgood{constructor de movimiento} \textemph{implícito}.
      \item Se genera un \textgood{operador de asignación de movimiento} \textemph{implícito}.
    \end{itemize}

  \mode<presentation>{\vfill\pause}
  \item Las operaciones implícitas se generan a partir de las correspondientes operaciones
        de cada miembro.
\end{itemize}
\end{frame}

\begin{frame}[t,fragile]{Simplifcando los tipos compuestos}
\begin{block}{Tipos compuestos}
\begin{lstlisting}
class muestra {
public:
  //...
  muestra(int id, const vecnum & datos) : id_{id}, datos_{datos} {}
  // Copia, movimiento y destrucción generadas por el compilador
private:
  std::string id_;
  vecnum datos_;
};
\end{lstlisting}
\end{block}
\end{frame}

\begin{frame}[t]{Funciones miembro especiales por defecto}
\begin{itemize}
  \item Las siguientes \textmark{funciones miembro especiales} se pueden forzar
        para que se \textemph{generen por defecto}.
    \begin{itemize}
      \item \textgood{Constructor de copia}.
      \item \textgood{Operador de asignación de copia}.
      \item \textgood{Constructor de movimiento}.
      \item \textgood{Operador de asignación de movimiento}.
      \item \textgood{Destructor}.
    \end{itemize}

  \mode<presentation>{\vfill\pause}
  \item Útil para añdirlas individualmente cuando no se generan implícitamente.

  \mode<presentation>{\vfill\pause}
  \item \textmark{Recuerda}:
    \begin{itemize}
      \item Si se declara alguna, habrá que declararlas todas.
      \item Si algún dato miembro tiene una función eliminada la correspondiente función
            no se puede generar.
    \end{itemize}
\end{itemize}
\end{frame}


\begin{frame}[t,fragile]{Funciones miembro especiales por defecto}
\begin{block}{Tipos compuestos}
\begin{lstlisting}
class muestra {
public:
  //...
  muestra(int id, const vecnum & datos) : id_{id}, datos_{datos} {}
  ~muestra() { std::cerr << "Muestra " << id_ << " destruida\n"; }
  muestra(muestra &&) = default;
private:
  std::string id_;
  vecnum datos_;
};
\end{lstlisting}
\end{block}

\begin{itemize}
  \item \cppid{muestra} \textemph{se puede} construir por movimiento.
  \item \cppid{muestra} \textbad{no se puede} asignar por movimiento.
\end{itemize}
\end{frame}

