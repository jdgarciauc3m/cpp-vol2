\section{Implementación de movimiento}

\begin{frame}[t,fragile]{Constructor de movimiento}
\begin{itemize}
  \item El \textgood{constructor de movimiento} es un constructor que se invoca
        cuando se construye un objeto a partir de:
    \begin{itemize}
      \item Un objeto temporal.
\begin{lstlisting}
vecnum v{obten_vecnum()};
vecnum v = obten_vecnum();
\end{lstlisting}
      \item El resultado de \cppid{std::move()}.
\begin{lstlisting}
vecnum v{std::move(w)};
vecnum v = std::move(w);
\end{lstlisting}
    \end{itemize}

  \mode<presentation>{\vfill\pause}
  \item Declaración del \textgood{constructor de movimiento}.
\begin{lstlisting}
vecnum(vecnum && v) noexcept;
\end{lstlisting}
    \begin{itemize}
      \item Parámetro de tipo \textmark{referencia a r-valor} (\cppkey{\&\&}).
      \item Es recomendable que sea \cppkey{noexcept}.
    \end{itemize}
\end{itemize}
\end{frame}

\begin{frame}[t,fragile]{Definición del constructor de movimiento}
\begin{itemize}
  \item \textmark{Pasos}:
    \begin{enumerate}
      \item El destino adquiere los recursos del origen.
      \item El origen se dejan en un estado válido que permita la destrucción.
    \end{enumerate}
\end{itemize}

\begin{columns}[T]

\pause
\column{.5\textwidth}
\begin{block}{Declaración}
\begin{lstlisting}
class vecnum {
public:
  //...
  vecnum(vecnum && v) noexcept;
private:
  int tamanyo_;
  double * buffer_;
};
\end{lstlisting}
\end{block}

\pause
\column{.5\textwidth}
\begin{block}{Definición}
\begin{lstlisting}
vecnum::vecnum(vecnum && v) noexcept
    : tamanyo_{v.tamanyo_},
      buffer_{v.buffer_}
{
  v.tamanyo_ = 0;
  v.buffer_ = nullptr;
}
\end{lstlisting}
\end{block}

\end{columns}
\end{frame}

\begin{frame}[t,fragile]{Tipos no movibles por construcción}
\begin{itemize}
  \item Si se quiere evitar que los objetos de un tipo se puedan mover en construcción,
        se puede \textbad{eliminar} el \textgood{constructor de movimiento}.
\begin{lstlisting}
class vecnum {
public:
  vecnum(vecnum &&) = delete;
  //...
};
\end{lstlisting}

  \mode<presentation>{\vfill\pause}
  \item Cualquier invocación \textmark{implícita} o \textmark{explícita}
        del constructor de movimiento provoca un \textbad{error de compilación}.
\begin{lstlisting}
vecnum make1() { return vecnum{1,2,3}; } // Elusión de copia/movimiento

vecnum make2() {
  vecnum r{1,2,3};
  return r; // Error: Constructor de movimiento eliminado
}
\end{lstlisting}

\end{itemize}
\end{frame}

\begin{frame}[t,fragile]{Asignación de movimiento}
\begin{itemize}
  \item El \textgood{operador de asignación de movimiento} se invoca cuando
        se asigna a un objeto:
    \begin{itemize}
      \item Un objeto temporal.
\begin{lstlisting}
v = obten_vecnum();
\end{lstlisting}

      \item El resultado de \cppid{std::move()}.
\begin{lstlisting}
v = std::move(w);
\end{lstlisting}
    \end{itemize}

  \mode<presentation>{\vfill\pause}
  \item Declaración del \textgood{operador de asignación de movimiento}.
\begin{lstlisting}
vecnum & operator=(vecnum && v) noexcept;
\end{lstlisting}
    \begin{itemize}
      \item Parámetro de tipo \textmark{referencia a r-valor} (\cppkey{\&\&}).
      \item Es recomendable que sea \cppkey{noexcept}.
    \end{itemize}
\end{itemize}
\end{frame}

\begin{frame}[t,fragile]{Definición de la asignación de movimiento}
\begin{itemize}
  \item \textmark{Pasos}:
    \begin{itemize}
      \item El destino libera sus recursos.
      \item El destino adquiere los recursos del origen.
      \item El origen se deja en un estado válido que permita la destrucción.
    \end{itemize}
\end{itemize}	

\mode<presentation>{\vfill\pause}
\begin{columns}[T]

\column{.5\textwidth}
\begin{block}{Declaración}
\begin{lstlisting}
class vecnum {
public:
  //...
  vecnum & operator=(vecnum && v) noexcept;
private:
  int tamanyo_;
  double * buffer_;
};
\end{lstlisting}
\end{block}

\pause
\column{.5\textwidth}
\begin{block}{Definición}
\begin{lstlisting}
vecnum & vecnum::operator=(vecnum && v) noexcept {
  if (this == & v) return *this;
  delete []buffer_;
  tamanyo_ = v.tamanyo_;
  buffer_ = v.buffer_;
  v.tamanyo_ = 0;
  v.buffer_ = nullptr;
  return *this;
}
\end{lstlisting}
\end{block}

\end{columns}
\end{frame}

\begin{frame}[t,fragile]{Tipos no movibles por asignación}
\begin{itemize}
  \item Si se quiere evitar que los objetos de un tipo se puedan mover en asignaciones,
       se puede \textbad{eliminar} el \textgood{operador de asignación de movimiento}.
\begin{lstlisting}
class vecnum {
public:
  vecnum & operator=(vecnum &&) = delete;
};
\end{lstlisting}

  \mode<presentation>{\vfill\pause}
  \item Cualquier invocación al operador de asignación de movimiento provoca un
        \textbad{error de compilación}.
\begin{lstlisting}
vecnum make() { return vecnum{1,2,3}; }

vecnum v;
v = make(); // Error: Operador de asignación eliminado
v = std::move(w); // Error: Operador de asignación eliminado
\end{lstlisting}
\end{itemize}
\end{frame}
